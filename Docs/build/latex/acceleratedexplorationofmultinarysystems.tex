%% Generated by Sphinx.
\def\sphinxdocclass{report}
\documentclass[letterpaper,10pt,english]{sphinxmanual}
\ifdefined\pdfpxdimen
   \let\sphinxpxdimen\pdfpxdimen\else\newdimen\sphinxpxdimen
\fi \sphinxpxdimen=.75bp\relax
\ifdefined\pdfimageresolution
    \pdfimageresolution= \numexpr \dimexpr1in\relax/\sphinxpxdimen\relax
\fi
%% let collapsible pdf bookmarks panel have high depth per default
\PassOptionsToPackage{bookmarksdepth=5}{hyperref}

\PassOptionsToPackage{warn}{textcomp}
\usepackage[utf8]{inputenc}
\ifdefined\DeclareUnicodeCharacter
% support both utf8 and utf8x syntaxes
  \ifdefined\DeclareUnicodeCharacterAsOptional
    \def\sphinxDUC#1{\DeclareUnicodeCharacter{"#1}}
  \else
    \let\sphinxDUC\DeclareUnicodeCharacter
  \fi
  \sphinxDUC{00A0}{\nobreakspace}
  \sphinxDUC{2500}{\sphinxunichar{2500}}
  \sphinxDUC{2502}{\sphinxunichar{2502}}
  \sphinxDUC{2514}{\sphinxunichar{2514}}
  \sphinxDUC{251C}{\sphinxunichar{251C}}
  \sphinxDUC{2572}{\textbackslash}
\fi
\usepackage{cmap}
\usepackage[T1]{fontenc}
\usepackage{amsmath,amssymb,amstext}
\usepackage{babel}



\usepackage{tgtermes}
\usepackage{tgheros}
\renewcommand{\ttdefault}{txtt}



\usepackage[Bjarne]{fncychap}
\usepackage{sphinx}

\fvset{fontsize=auto}
\usepackage{geometry}


% Include hyperref last.
\usepackage{hyperref}
% Fix anchor placement for figures with captions.
\usepackage{hypcap}% it must be loaded after hyperref.
% Set up styles of URL: it should be placed after hyperref.
\urlstyle{same}

\addto\captionsenglish{\renewcommand{\contentsname}{Contents:}}

\usepackage{sphinxmessages}
\setcounter{tocdepth}{1}



\title{Accelerated exploration of multinary systems}
\date{Feb 15, 2022}
\release{1.1}
\author{Elise Garel, Jean\sphinxhyphen{}Luc Parouty}
\newcommand{\sphinxlogo}{\vbox{}}
\renewcommand{\releasename}{Release}
\makeindex
\begin{document}

\pagestyle{empty}
\sphinxmaketitle
\pagestyle{plain}
\sphinxtableofcontents
\pagestyle{normal}
\phantomsection\label{\detokenize{index::doc}}



\chapter{ExperimentsPlannification}
\label{\detokenize{ExperimentsPlannification:experimentsplannification}}\label{\detokenize{ExperimentsPlannification::doc}}\phantomsection\label{\detokenize{ExperimentsPlannification:module-modules}}\index{modules (module)@\spxentry{modules}\spxextra{module}}\index{check\_do\_not\_align() (in module modules)@\spxentry{check\_do\_not\_align()}\spxextra{in module modules}}

\begin{fulllineitems}
\phantomsection\label{\detokenize{ExperimentsPlannification:modules.check_do_not_align}}\pysiglinewithargsret{\sphinxcode{\sphinxupquote{modules.}}\sphinxbfcode{\sphinxupquote{check\_do\_not\_align}}}{\emph{name\_alignments}, \emph{index}, \emph{do\_not\_align}}{}
\sphinxAtStartPar
Check user condition to not align certain mixtures in the same gradient
\begin{quote}\begin{description}
\item[{Parameters}] \leavevmode\begin{itemize}
\item {} 
\sphinxAtStartPar
\sphinxstyleliteralstrong{\sphinxupquote{name\_alignments}} (\sphinxcode{\sphinxupquote{array(str)}}) \textendash{} name of points through which the gradients passes

\item {} 
\sphinxAtStartPar
\sphinxstyleliteralstrong{\sphinxupquote{index}} (\sphinxcode{\sphinxupquote{int}}) \textendash{} index of alignments in the list of all alignments

\item {} 
\sphinxAtStartPar
\sphinxstyleliteralstrong{\sphinxupquote{do\_not\_align}} (\sphinxcode{\sphinxupquote{cell(list(str))}}) \textendash{} list of mixtures that must not be aligned

\end{itemize}

\item[{Returns}] \leavevmode
\sphinxAtStartPar
indicator: “ok” if alignement respects the user condition; else return “not ok”.

\item[{Return type}] \leavevmode
\sphinxAtStartPar
str

\end{description}\end{quote}

\end{fulllineitems}

\phantomsection\label{\detokenize{ExperimentsPlannification:module-modules}}\index{modules (module)@\spxentry{modules}\spxextra{module}}\index{check\_not\_repeat() (in module modules)@\spxentry{check\_not\_repeat()}\spxextra{in module modules}}

\begin{fulllineitems}
\phantomsection\label{\detokenize{ExperimentsPlannification:modules.check_not_repeat}}\pysiglinewithargsret{\sphinxcode{\sphinxupquote{modules.}}\sphinxbfcode{\sphinxupquote{check\_not\_repeat}}}{\emph{name\_alignments}, \emph{index}, \emph{name\_alignement\_opt}, \emph{not\_repeat}}{}
\sphinxAtStartPar
Check user condition to not repeat certain mixtures in gradients set
\begin{quote}\begin{description}
\item[{Parameters}] \leavevmode\begin{itemize}
\item {} 
\sphinxAtStartPar
\sphinxstyleliteralstrong{\sphinxupquote{name\_alignments}} (\sphinxcode{\sphinxupquote{array(str)}}) \textendash{} name of points through which the gradients passes

\item {} 
\sphinxAtStartPar
\sphinxstyleliteralstrong{\sphinxupquote{index}} (\sphinxcode{\sphinxupquote{int}}) \textendash{} index of alignments in the list of all alignments

\item {} 
\sphinxAtStartPar
\sphinxstyleliteralstrong{\sphinxupquote{name\_alignement\_opt}} (\sphinxcode{\sphinxupquote{array(str)}}) \textendash{} gradients set that are already selected.

\item {} 
\sphinxAtStartPar
\sphinxstyleliteralstrong{\sphinxupquote{not\_repeat}} (\sphinxcode{\sphinxupquote{array(str)}}) \textendash{} mixtures to no repeat in the gradients set

\end{itemize}

\item[{Returns}] \leavevmode
\sphinxAtStartPar
indicator: “ok” if alignement respects the user condition; else return “not ok”.

\item[{Return type}] \leavevmode
\sphinxAtStartPar
str

\end{description}\end{quote}

\end{fulllineitems}

\phantomsection\label{\detokenize{ExperimentsPlannification:module-modules}}\index{modules (module)@\spxentry{modules}\spxextra{module}}\index{check\_repeat\_only() (in module modules)@\spxentry{check\_repeat\_only()}\spxextra{in module modules}}

\begin{fulllineitems}
\phantomsection\label{\detokenize{ExperimentsPlannification:modules.check_repeat_only}}\pysiglinewithargsret{\sphinxcode{\sphinxupquote{modules.}}\sphinxbfcode{\sphinxupquote{check\_repeat\_only}}}{\emph{name\_alignments}, \emph{index}, \emph{name\_alignement\_opt}, \emph{repeat\_only}}{}
\sphinxAtStartPar
Check user condition to not repeat certain mixtures in gradients set
\begin{quote}\begin{description}
\item[{Parameters}] \leavevmode\begin{itemize}
\item {} 
\sphinxAtStartPar
\sphinxstyleliteralstrong{\sphinxupquote{name\_alignments}} (\sphinxcode{\sphinxupquote{array(str)}}) \textendash{} name of points through which the gradients passes

\item {} 
\sphinxAtStartPar
\sphinxstyleliteralstrong{\sphinxupquote{index}} (\sphinxcode{\sphinxupquote{int}}) \textendash{} index of alignments in the list of all alignments

\item {} 
\sphinxAtStartPar
\sphinxstyleliteralstrong{\sphinxupquote{name\_alignement\_opt}} (\sphinxcode{\sphinxupquote{array(str)}}) \textendash{} gradients set that are allready selected.

\item {} 
\sphinxAtStartPar
\sphinxstyleliteralstrong{\sphinxupquote{repeat\_only}} (\sphinxcode{\sphinxupquote{array(str,int)}}) \textendash{} name of mixtures that must be repeated a limited number of time and this limited number of time

\end{itemize}

\item[{Returns}] \leavevmode
\sphinxAtStartPar
indicator: “ok” if alignement respects the user condition; else return “not ok”.

\item[{Return type}] \leavevmode
\sphinxAtStartPar
str

\end{description}\end{quote}

\end{fulllineitems}

\phantomsection\label{\detokenize{ExperimentsPlannification:module-modules}}\index{modules (module)@\spxentry{modules}\spxextra{module}}\index{compute\_alignments() (in module modules)@\spxentry{compute\_alignments()}\spxextra{in module modules}}

\begin{fulllineitems}
\phantomsection\label{\detokenize{ExperimentsPlannification:modules.compute_alignments}}\pysiglinewithargsret{\sphinxcode{\sphinxupquote{modules.}}\sphinxbfcode{\sphinxupquote{compute\_alignments}}}{\emph{mixture}, \emph{name\_mixture}, \emph{nb\_type\_mixture}}{}
\sphinxAtStartPar
For a reference mixture, the function calculates the vector coefficient between this reference mixture and all the other mixtures with same or higher order. Then it looks for  equals vector coefficients for segments with a common point to determine which two other mixture points are aligned with the reference mixture
\begin{quote}\begin{description}
\item[{Parameters}] \leavevmode\begin{itemize}
\item {} 
\sphinxAtStartPar
\sphinxstyleliteralstrong{\sphinxupquote{mixture}} (\sphinxcode{\sphinxupquote{cell\{array\}}}) \textendash{} coordinates of mixtures, cell index being the mixture order \sphinxstyleemphasis{eg:mixtures\{2\} contains the binaries coordinates}

\item {} 
\sphinxAtStartPar
\sphinxstyleliteralstrong{\sphinxupquote{name\_mixture}} (\sphinxcode{\sphinxupquote{cell\{str\}}}) \textendash{} name of mixture, cell index being the mixture order

\item {} 
\sphinxAtStartPar
\sphinxstyleliteralstrong{\sphinxupquote{nb\_type\_mixture}} (\sphinxcode{\sphinxupquote{int}}) \textendash{} number of type/order of mixtures to explore

\end{itemize}

\item[{Returns}] \leavevmode
\sphinxAtStartPar
\begin{itemize}
\item {} 
\sphinxAtStartPar
alignments : coordinates of the mixtures through which the gradient pass (3x3 columns)

\item {} 
\sphinxAtStartPar
name\_alignments: mixture names through which the gradient pass

\end{itemize}


\item[{Return type}] \leavevmode
\sphinxAtStartPar
array(float),array(str)

\end{description}\end{quote}

\end{fulllineitems}

\phantomsection\label{\detokenize{ExperimentsPlannification:module-modules}}\index{modules (module)@\spxentry{modules}\spxextra{module}}\index{compute\_planes() (in module modules)@\spxentry{compute\_planes()}\spxextra{in module modules}}

\begin{fulllineitems}
\phantomsection\label{\detokenize{ExperimentsPlannification:modules.compute_planes}}\pysiglinewithargsret{\sphinxcode{\sphinxupquote{modules.}}\sphinxbfcode{\sphinxupquote{compute\_planes}}}{\emph{name\_alignment}, \emph{alignments}, \emph{nb\_type\_mixture}}{}
\sphinxAtStartPar
From the gradients, planes are defines in the composition space made by 3 gradients with common points, encompassing 7 points of the mixtrure design. This means that the plane is centered on one of the point of the mixture design
\begin{quote}

\sphinxAtStartPar
\sphinxstyleemphasis{eg: Nb\sphinxhyphen{}NbTi\sphinxhyphen{}Ti, Ti\sphinxhyphen{}TiZr\sphinxhyphen{}Zr and Nb\sphinxhyphen{}NbZr\sphinxhyphen{}Zr are forming a plane in a
compositional space center on the ternary NbTiZr wich is a point of the
mixture design: the plane is valid}
\end{quote}
\begin{quote}\begin{description}
\item[{Parameters}] \leavevmode\begin{itemize}
\item {} 
\sphinxAtStartPar
\sphinxstyleliteralstrong{\sphinxupquote{name\_alignment}} (\sphinxcode{\sphinxupquote{array(str)}}) \textendash{} points through which the gradient go

\item {} 
\sphinxAtStartPar
\sphinxstyleliteralstrong{\sphinxupquote{alignments}} (\sphinxcode{\sphinxupquote{array(float)}}) \textendash{} coordinates of the points through which the gradient go (3x3 columns)

\item {} 
\sphinxAtStartPar
\sphinxstyleliteralstrong{\sphinxupquote{nb\_type\_mixture}} (\sphinxcode{\sphinxupquote{int}}) \textendash{} number of type/order of mixtures to explore

\end{itemize}

\item[{Returns}] \leavevmode
\sphinxAtStartPar
plane\_points: mixture names encompassed by the planes

\item[{Returns}] \leavevmode
\sphinxAtStartPar
plane\_coord: coordinates of the mixtures encompassed by the planes (7x3 columns)

\end{description}\end{quote}

\end{fulllineitems}

\phantomsection\label{\detokenize{ExperimentsPlannification:module-modules}}\index{modules (module)@\spxentry{modules}\spxextra{module}}\index{coordinates\_name\_centroid\_points() (in module modules)@\spxentry{coordinates\_name\_centroid\_points()}\spxextra{in module modules}}

\begin{fulllineitems}
\phantomsection\label{\detokenize{ExperimentsPlannification:modules.coordinates_name_centroid_points}}\pysiglinewithargsret{\sphinxcode{\sphinxupquote{modules.}}\sphinxbfcode{\sphinxupquote{coordinates\_name\_centroid\_points}}}{\emph{nb\_elements}, \emph{name\_elements}}{}
\sphinxAtStartPar
From the number and the name of the system elements, the function
calculates the coordinates of the pure elements (standard uniform
distribution in space) and of the equiolar mixtures of the Simplex
Centroide mixture Design (all binaries, ternairies…).
\begin{quote}\begin{description}
\item[{Parameters}] \leavevmode
\sphinxAtStartPar
\sphinxstyleliteralstrong{\sphinxupquote{nb\_elements}} (\sphinxcode{\sphinxupquote{int}}) \textendash{} number of components

\item[{Name\_element list(str)}] \leavevmode
\sphinxAtStartPar
namer of components

\item[{Returns}] \leavevmode
\sphinxAtStartPar
mixture: cell coordinates of all equimolar mixture

\item[{Return type}] \leavevmode
\sphinxAtStartPar
cell

\item[{Returns}] \leavevmode
\sphinxAtStartPar
name\_mixture: containing the names of the equimolar mixtures.

\item[{Return type}] \leavevmode
\sphinxAtStartPar
cell

\end{description}\end{quote}

\end{fulllineitems}

\phantomsection\label{\detokenize{ExperimentsPlannification:module-modules}}\index{modules (module)@\spxentry{modules}\spxextra{module}}\index{count\_occur() (in module modules)@\spxentry{count\_occur()}\spxextra{in module modules}}

\begin{fulllineitems}
\phantomsection\label{\detokenize{ExperimentsPlannification:modules.count_occur}}\pysiglinewithargsret{\sphinxcode{\sphinxupquote{modules.}}\sphinxbfcode{\sphinxupquote{count\_occur}}}{\emph{element}, \emph{list}}{}
\sphinxAtStartPar
Count the numer of occurence of an element in a list
\begin{quote}\begin{description}
\item[{Parameters}] \leavevmode\begin{itemize}
\item {} 
\sphinxAtStartPar
\sphinxstyleliteralstrong{\sphinxupquote{element}} \textendash{} counted number or string

\item {} 
\sphinxAtStartPar
\sphinxstyleliteralstrong{\sphinxupquote{list}} (\sphinxcode{\sphinxupquote{list}}) \textendash{} list in which the element is counted

\end{itemize}

\item[{Returns}] \leavevmode
\sphinxAtStartPar
count: number of repetition of the element in list

\item[{Return type}] \leavevmode
\sphinxAtStartPar
int

\end{description}\end{quote}

\end{fulllineitems}

\phantomsection\label{\detokenize{ExperimentsPlannification:module-modules}}\index{modules (module)@\spxentry{modules}\spxextra{module}}\index{fix\_nb\_repetition() (in module modules)@\spxentry{fix\_nb\_repetition()}\spxextra{in module modules}}

\begin{fulllineitems}
\phantomsection\label{\detokenize{ExperimentsPlannification:modules.fix_nb_repetition}}\pysiglinewithargsret{\sphinxcode{\sphinxupquote{modules.}}\sphinxbfcode{\sphinxupquote{fix\_nb\_repetition}}}{\emph{repeat\_list}, \emph{fig}, \emph{position}}{}\begin{description}
\item[{This function is a callbacks of push buttons associated to listboxes}] \leavevmode
\sphinxAtStartPar
When the buttons are pushed, the function identifies which mixture should be repeated
Then it display in the interface the names of the mixtures that should be
repeated and an edit box in which the user can enter the number of repetitions.

\end{description}
\begin{quote}\begin{description}
\item[{Parameters}] \leavevmode\begin{itemize}
\item {} 
\sphinxAtStartPar
\sphinxstyleliteralstrong{\sphinxupquote{repeat\_list}} (\sphinxcode{\sphinxupquote{UIcontrol}}) \textendash{} contains the mixture that should be repeated

\item {} 
\sphinxAtStartPar
\sphinxstyleliteralstrong{\sphinxupquote{fig}} (\sphinxcode{\sphinxupquote{figure}}) \textendash{} working interface / window

\item {} 
\sphinxAtStartPar
\sphinxstyleliteralstrong{\sphinxupquote{position}} (\sphinxcode{\sphinxupquote{list(float)}}) \textendash{} position features of repeated list

\end{itemize}

\item[{Returns}] \leavevmode
\sphinxAtStartPar
nb\_repet: edit boxes in which the user will enter the number of repetition of each mixture

\end{description}\end{quote}

\end{fulllineitems}

\phantomsection\label{\detokenize{ExperimentsPlannification:module-modules}}\index{modules (module)@\spxentry{modules}\spxextra{module}}\index{get\_elements() (in module modules)@\spxentry{get\_elements()}\spxextra{in module modules}}

\begin{fulllineitems}
\phantomsection\label{\detokenize{ExperimentsPlannification:modules.get_elements}}\pysiglinewithargsret{\sphinxcode{\sphinxupquote{modules.}}\sphinxbfcode{\sphinxupquote{get\_elements}}}{\emph{elements}, \emph{fig1}}{}
\sphinxAtStartPar
Acquire the components name entered by the user
\begin{quote}\begin{description}
\item[{Parameters}] \leavevmode\begin{itemize}
\item {} 
\sphinxAtStartPar
\sphinxstyleliteralstrong{\sphinxupquote{elements}} (\sphinxcode{\sphinxupquote{UIcontrol}}) \textendash{} edit boxes in which the user has entered the elements names

\item {} 
\sphinxAtStartPar
\sphinxstyleliteralstrong{\sphinxupquote{fig1}} (\sphinxcode{\sphinxupquote{figure}}) \textendash{} interface window

\end{itemize}

\item[{Returns}] \leavevmode
\sphinxAtStartPar
name\_elements: name of elements

\item[{Return type}] \leavevmode
\sphinxAtStartPar
list(str)

\end{description}\end{quote}

\end{fulllineitems}

\phantomsection\label{\detokenize{ExperimentsPlannification:module-modules}}\index{modules (module)@\spxentry{modules}\spxextra{module}}\index{gradients\_set() (in module modules)@\spxentry{gradients\_set()}\spxextra{in module modules}}

\begin{fulllineitems}
\phantomsection\label{\detokenize{ExperimentsPlannification:modules.gradients_set}}\pysiglinewithargsret{\sphinxcode{\sphinxupquote{modules.}}\sphinxbfcode{\sphinxupquote{gradients\_set}}}{\emph{name\_mixture}, \emph{mixture}, \emph{alignments}, \emph{name\_alignement}}{}\begin{description}
\item[{Selection of a gradients set that pass at least once through each point}] \leavevmode
\sphinxAtStartPar
of the mixture design and that respect user condition inputs

\end{description}
\begin{quote}\begin{description}
\item[{Parameters}] \leavevmode\begin{itemize}
\item {} 
\sphinxAtStartPar
\sphinxstyleliteralstrong{\sphinxupquote{name\_mixture}} (\sphinxcode{\sphinxupquote{cell(str)}}) \textendash{} name of mixture, cell index being the mixture order

\item {} 
\sphinxAtStartPar
\sphinxstyleliteralstrong{\sphinxupquote{mixture}} (\sphinxcode{\sphinxupquote{cell(float)}}) \textendash{} coordinates of mixtures, cell index being the mixture order \sphinxstyleemphasis{eg:mixtures\{2\} contains the binaries coordinates}

\end{itemize}

\end{description}\end{quote}

\sphinxAtStartPar
:param cell(float) alignments:coordinates of the points through which the gradient pass (3x3 columns)
:param cell(str) name\_alignement: points through which the gradient pass
:return array(str) name\_alignement\_opt: name of mixture trhough whch the set of gradients pass
:return array(float) alignement\_opt: coordinates of mixture trhough whch the set of gradients pass

\end{fulllineitems}

\phantomsection\label{\detokenize{ExperimentsPlannification:module-modules}}\index{modules (module)@\spxentry{modules}\spxextra{module}}\index{index\_alignments() (in module modules)@\spxentry{index\_alignments()}\spxextra{in module modules}}

\begin{fulllineitems}
\phantomsection\label{\detokenize{ExperimentsPlannification:modules.index_alignments}}\pysiglinewithargsret{\sphinxcode{\sphinxupquote{modules.}}\sphinxbfcode{\sphinxupquote{index\_alignments}}}{\emph{cell\_coeff\_dir}}{}\begin{description}
\item[{Called in {\color{red}\bfseries{}\textasciigrave{}compute\_alignments\textasciigrave{}\_}: we get cell structure with vector coefficient between one reference mixture and all the mixtures with the same or higher orders .}] \leavevmode
\sphinxAtStartPar
This function compares all the coefficients one by one to find equal ones

\end{description}
\begin{quote}\begin{description}
\item[{Parameters}] \leavevmode
\sphinxAtStartPar
\sphinxstyleliteralstrong{\sphinxupquote{cell\_coeff\_dir}} (\sphinxcode{\sphinxupquote{cell}}) \textendash{} contains director coefficient of vectors between one reference mixtures and all the others with same or higher order.

\item[{Returns}] \leavevmode
\sphinxAtStartPar
cell indice\_cell, indice\_list: indices of the cell and list where two coefficients are equals. Allow to identify pair of equal coefficient to identify aligned points.

\end{description}\end{quote}

\end{fulllineitems}

\phantomsection\label{\detokenize{ExperimentsPlannification:module-modules}}\index{modules (module)@\spxentry{modules}\spxextra{module}}\index{kill\_program() (in module modules)@\spxentry{kill\_program()}\spxextra{in module modules}}

\begin{fulllineitems}
\phantomsection\label{\detokenize{ExperimentsPlannification:modules.kill_program}}\pysiglinewithargsret{\sphinxcode{\sphinxupquote{modules.}}\sphinxbfcode{\sphinxupquote{kill\_program}}}{}{}
\sphinxAtStartPar
Kill the programis the user pushed STOP button
\begin{quote}\begin{description}
\item[{Returns}] \leavevmode
\sphinxAtStartPar
display the message “kill” to indicate state

\end{description}\end{quote}

\end{fulllineitems}

\phantomsection\label{\detokenize{ExperimentsPlannification:module-modules}}\index{modules (module)@\spxentry{modules}\spxextra{module}}\index{lineIntersect3D() (in module modules)@\spxentry{lineIntersect3D()}\spxextra{in module modules}}

\begin{fulllineitems}
\phantomsection\label{\detokenize{ExperimentsPlannification:modules.lineIntersect3D}}\pysiglinewithargsret{\sphinxcode{\sphinxupquote{modules.}}\sphinxbfcode{\sphinxupquote{lineIntersect3D}}}{\emph{PA}, \emph{PB}}{}
\sphinxAtStartPar
Find intersection point of lines in 3D space, in the least squares sense.
\begin{quote}\begin{description}
\item[{Parameters}] \leavevmode\begin{itemize}
\item {} 
\sphinxAtStartPar
\sphinxstyleliteralstrong{\sphinxupquote{PA}} \textendash{} Nx3\sphinxhyphen{}matrix containing starting point of N lines

\item {} 
\sphinxAtStartPar
\sphinxstyleliteralstrong{\sphinxupquote{PB}} \textendash{} Nx3\sphinxhyphen{}matrix containing end point of N lines

\end{itemize}

\item[{Returns}] \leavevmode
\sphinxAtStartPar
P\_Intersect: Best intersection point of the N lines, in least squares sense.

\item[{Returns}] \leavevmode
\sphinxAtStartPar
distances: Distances from intersection point to the input lines

\end{description}\end{quote}

\sphinxAtStartPar
Anders Eikenes (2022). Intersection point of lines in 3D space
(\sphinxurl{https://www.mathworks.com/matlabcentral/fileexchange/37192-intersection-point-of-lines-in-3d-space}), MATLAB Central File Exchange. Retrieved February 10, 2022.

\end{fulllineitems}

\phantomsection\label{\detokenize{ExperimentsPlannification:module-modules}}\index{modules (module)@\spxentry{modules}\spxextra{module}}\index{nb\_repetitions() (in module modules)@\spxentry{nb\_repetitions()}\spxextra{in module modules}}

\begin{fulllineitems}
\phantomsection\label{\detokenize{ExperimentsPlannification:modules.nb_repetitions}}\pysiglinewithargsret{\sphinxcode{\sphinxupquote{modules.}}\sphinxbfcode{\sphinxupquote{nb\_repetitions}}}{}{}
\end{fulllineitems}

\phantomsection\label{\detokenize{ExperimentsPlannification:module-modules}}\index{modules (module)@\spxentry{modules}\spxextra{module}}\index{listing\_targets() (in module modules)@\spxentry{listing\_targets()}\spxextra{in module modules}}

\begin{fulllineitems}
\phantomsection\label{\detokenize{ExperimentsPlannification:modules.listing_targets}}\pysiglinewithargsret{\sphinxcode{\sphinxupquote{modules.}}\sphinxbfcode{\sphinxupquote{listing\_targets}}}{\emph{name\_alignement\_opt}}{}
\sphinxAtStartPar
Lists the targets to use from the selected optimized set of gradients
\begin{quote}\begin{description}
\item[{Parameters}] \leavevmode
\sphinxAtStartPar
\sphinxstyleliteralstrong{\sphinxupquote{name\_alignement\_opt}} (\sphinxcode{\sphinxupquote{array(str)}}) \textendash{} name of mixtures through which pass the gradients

\item[{Returns}] \leavevmode
\sphinxAtStartPar
list(str) list\_target: list the target compositions to use to deposit these gradients

\end{description}\end{quote}

\end{fulllineitems}

\phantomsection\label{\detokenize{ExperimentsPlannification:module-modules}}\index{modules (module)@\spxentry{modules}\spxextra{module}}\index{listing\_targets\_3cath() (in module modules)@\spxentry{listing\_targets\_3cath()}\spxextra{in module modules}}

\begin{fulllineitems}
\phantomsection\label{\detokenize{ExperimentsPlannification:modules.listing_targets_3cath}}\pysiglinewithargsret{\sphinxcode{\sphinxupquote{modules.}}\sphinxbfcode{\sphinxupquote{listing\_targets\_3cath}}}{\emph{name\_planes\_opt}}{}\begin{description}
\item[{Lists the targets to use from the selected optimized set of planar}] \leavevmode
\sphinxAtStartPar
gradients

\end{description}
\begin{quote}\begin{description}
\item[{Parameters}] \leavevmode
\sphinxAtStartPar
\sphinxstyleliteralstrong{\sphinxupquote{name\_planes\_opt}} (\sphinxcode{\sphinxupquote{array(str)}}) \textendash{} name of mixtures encompassed by planar gradients

\item[{Returns}] \leavevmode
\sphinxAtStartPar
list(str) list\_target: list the target compositions to use to deposit these gradients

\end{description}\end{quote}

\end{fulllineitems}

\phantomsection\label{\detokenize{ExperimentsPlannification:module-modules}}\index{modules (module)@\spxentry{modules}\spxextra{module}}\index{plot\_compo\_space\_gradients() (in module modules)@\spxentry{plot\_compo\_space\_gradients()}\spxextra{in module modules}}

\begin{fulllineitems}
\phantomsection\label{\detokenize{ExperimentsPlannification:modules.plot_compo_space_gradients}}\pysiglinewithargsret{\sphinxcode{\sphinxupquote{modules.}}\sphinxbfcode{\sphinxupquote{plot\_compo\_space\_gradients}}}{\emph{nb\_elements}, \emph{mixture}, \emph{name\_mixture}, \emph{name\_elements}, \emph{gradients}, \emph{gradients\_color}}{}
\sphinxAtStartPar
Plot the composition space with all the simplexe centroid points and linear gradients
\begin{quote}\begin{description}
\item[{Parameters}] \leavevmode\begin{itemize}
\item {} 
\sphinxAtStartPar
\sphinxstyleliteralstrong{\sphinxupquote{nb\_elements}} (\sphinxcode{\sphinxupquote{int}}) \textendash{} number of components

\item {} 
\sphinxAtStartPar
\sphinxstyleliteralstrong{\sphinxupquote{mixture}} (\sphinxcode{\sphinxupquote{cell(float)}}) \textendash{} mixture points coordinates

\item {} 
\sphinxAtStartPar
\sphinxstyleliteralstrong{\sphinxupquote{name\_mixture}} (\sphinxcode{\sphinxupquote{cell(str)}}) \textendash{} mixture names

\end{itemize}

\end{description}\end{quote}

\sphinxAtStartPar
:param list(str) name\_elements:name of the components
:param array(str) gradients : coordinates of the gradients points
:param str/list(float) gradients\_color: color of the gradients for plot
:return: fig: plot the compositions space dans gradients

\end{fulllineitems}

\phantomsection\label{\detokenize{ExperimentsPlannification:module-modules}}\index{modules (module)@\spxentry{modules}\spxextra{module}}\index{plot\_compo\_space\_planes() (in module modules)@\spxentry{plot\_compo\_space\_planes()}\spxextra{in module modules}}

\begin{fulllineitems}
\phantomsection\label{\detokenize{ExperimentsPlannification:modules.plot_compo_space_planes}}\pysiglinewithargsret{\sphinxcode{\sphinxupquote{modules.}}\sphinxbfcode{\sphinxupquote{plot\_compo\_space\_planes}}}{\emph{nb\_elements}, \emph{mixture}, \emph{name\_mixture}, \emph{name\_elements}, \emph{plane\_coord}, \emph{plane\_color}, \emph{fignumber)\%position}}{}
\sphinxAtStartPar
Plot the composition space with all the simplexe centroid points and planar gradients
\begin{quote}\begin{description}
\item[{Parameters}] \leavevmode\begin{itemize}
\item {} 
\sphinxAtStartPar
\sphinxstyleliteralstrong{\sphinxupquote{nb\_elements}} (\sphinxcode{\sphinxupquote{int}}) \textendash{} number of components

\item {} 
\sphinxAtStartPar
\sphinxstyleliteralstrong{\sphinxupquote{mixture}} (\sphinxcode{\sphinxupquote{cell(float)}}) \textendash{} mixture points coordinates

\item {} 
\sphinxAtStartPar
\sphinxstyleliteralstrong{\sphinxupquote{name\_mixture}} (\sphinxcode{\sphinxupquote{cell(str)}}) \textendash{} mixture names

\end{itemize}

\end{description}\end{quote}

\sphinxAtStartPar
:param list(str) name\_elements:name of the components
:param array(str) plane\_coord : coordinates of the planes points
:param str/list(float) plane\_color: color of the plane for plot
:return: fig: plot the compositions space dans gradients

\end{fulllineitems}

\phantomsection\label{\detokenize{ExperimentsPlannification:module-modules}}\index{modules (module)@\spxentry{modules}\spxextra{module}}\index{parameters\_file() (in module modules)@\spxentry{parameters\_file()}\spxextra{in module modules}}

\begin{fulllineitems}
\phantomsection\label{\detokenize{ExperimentsPlannification:modules.parameters_file}}\pysiglinewithargsret{\sphinxcode{\sphinxupquote{modules.}}\sphinxbfcode{\sphinxupquote{parameters\_file}}}{}{}
\sphinxAtStartPar
Write the users inputs and chosen parameters for one run of the interface in text file.

\end{fulllineitems}

\phantomsection\label{\detokenize{ExperimentsPlannification:module-modules}}\index{modules (module)@\spxentry{modules}\spxextra{module}}\index{planes\_set() (in module modules)@\spxentry{planes\_set()}\spxextra{in module modules}}

\begin{fulllineitems}
\phantomsection\label{\detokenize{ExperimentsPlannification:modules.planes_set}}\pysiglinewithargsret{\sphinxcode{\sphinxupquote{modules.}}\sphinxbfcode{\sphinxupquote{planes\_set}}}{\emph{name\_mixture}, \emph{mixture}, \emph{planes}, \emph{name\_planes}}{}
\sphinxAtStartPar
Selection of a planes set that encompass at least once  each point of the mixture design and that respect user condition inputs
\begin{quote}\begin{description}
\item[{Parameters}] \leavevmode\begin{itemize}
\item {} 
\sphinxAtStartPar
\sphinxstyleliteralstrong{\sphinxupquote{name\_mixture}} (\sphinxcode{\sphinxupquote{cell(str)}}) \textendash{} name of mixture, cell index being the mixture order

\item {} 
\sphinxAtStartPar
\sphinxstyleliteralstrong{\sphinxupquote{mixture}} (\sphinxcode{\sphinxupquote{cell(float)}}) \textendash{} coordinates of mixtures, cell index being the mixture order \sphinxstyleemphasis{eg:mixtures\{2\} contains the binaries coordinates}

\end{itemize}

\end{description}\end{quote}

\sphinxAtStartPar
:param cell(float) plane:coordinates of the points through which the planes pass (3x3 columns)
:param cell(str) name\_planes: points through which the planes pass
:return: array(str) name\_planes\_opt: name of mixture trhough whch the set of planes pass
:return: array(float) planes\_opt: coordinates of mixture trhough whch the set of planes pass

\end{fulllineitems}

\phantomsection\label{\detokenize{ExperimentsPlannification:module-modules}}\index{modules (module)@\spxentry{modules}\spxextra{module}}\index{price\_calculation() (in module modules)@\spxentry{price\_calculation()}\spxextra{in module modules}}

\begin{fulllineitems}
\phantomsection\label{\detokenize{ExperimentsPlannification:modules.price_calculation}}\pysiglinewithargsret{\sphinxcode{\sphinxupquote{modules.}}\sphinxbfcode{\sphinxupquote{price\_calculation}}}{\emph{prices\_list}, \emph{target\_list}}{}
\sphinxAtStartPar
Calculate the price of a set of experiment
\begin{quote}\begin{description}
\item[{Parameters}] \leavevmode\begin{itemize}
\item {} 
\sphinxAtStartPar
\sphinxstyleliteralstrong{\sphinxupquote{prices\_list}} (\sphinxcode{\sphinxupquote{list(str,float)}}) \textendash{} list of possible targets and associated price

\item {} 
\sphinxAtStartPar
\sphinxstyleliteralstrong{\sphinxupquote{targets}} (\sphinxcode{\sphinxupquote{list(str)}}) \textendash{} list of targets associated to one set of linear gradients or planar gradients

\end{itemize}

\item[{Returns}] \leavevmode
\sphinxAtStartPar
price: total price of the targets required for a set of linear gradients or planar gradients

\item[{Return type}] \leavevmode
\sphinxAtStartPar
float

\end{description}\end{quote}

\end{fulllineitems}

\phantomsection\label{\detokenize{ExperimentsPlannification:module-modules}}\index{modules (module)@\spxentry{modules}\spxextra{module}}\index{vector\_coeff() (in module modules)@\spxentry{vector\_coeff()}\spxextra{in module modules}}

\begin{fulllineitems}
\phantomsection\label{\detokenize{ExperimentsPlannification:modules.vector_coeff}}\pysiglinewithargsret{\sphinxcode{\sphinxupquote{modules.}}\sphinxbfcode{\sphinxupquote{vector\_coeff}}}{\emph{A}, \emph{B}}{}
\sphinxAtStartPar
Compute normed vector coefficients between two points.
\begin{quote}\begin{description}
\item[{Parameters}] \leavevmode
\sphinxAtStartPar
\sphinxstyleliteralstrong{\sphinxupquote{A,B}} (\sphinxcode{\sphinxupquote{list(float)}}) \textendash{} coordinates of two points

\item[{Returns}] \leavevmode
\sphinxAtStartPar
coordinates of the normed vector corresponding to (AB) line

\end{description}\end{quote}

\end{fulllineitems}



\chapter{pyterk package}
\label{\detokenize{pyterk:pyterk-package}}\label{\detokenize{pyterk::doc}}

\section{Module contents}
\label{\detokenize{pyterk:module-pyterk}}\label{\detokenize{pyterk:module-contents}}\index{module@\spxentry{module}!pyterk@\spxentry{pyterk}}\index{pyterk@\spxentry{pyterk}!module@\spxentry{module}}
\sphinxAtStartPar
\sphinxstylestrong{PyTerK} \sphinxhyphen{} A Python Iterated K\sphinxhyphen{}fold cross validation with shuffling

\sphinxAtStartPar
By E Garel / JL Parouty \sphinxhyphen{} SIMaP 2021

\sphinxAtStartPar
This package allows you to perform a \sphinxstylestrong{statistical evaluation} of different learning strategies (Keras/sklearn) by varying different (hyper)parameters.

\sphinxAtStartPar
\#\# Description :

\sphinxAtStartPar
It is possible to combine the following (hyper)parameters :
\begin{itemize}
\item {} 
\sphinxAtStartPar
datasets

\item {} 
\sphinxAtStartPar
models (with their characteristics…)

\item {} 
\sphinxAtStartPar
batch size

\item {} 
\sphinxAtStartPar
epochs

\item {} 
\sphinxAtStartPar
iterations

\item {} 
\sphinxAtStartPar
k fold

\item {} 
\sphinxAtStartPar
seed (to control pseudo random generator)

\end{itemize}

\sphinxAtStartPar
It is possible, for example, to combine 3 datasets, with 3 models and to perform for each combination, 5 iterations of a cross validation of KFold type, with k=10.
In this case, the total number of models to test would be 3x3x5x10=450 training sessions…
So, be careful, the number of model.fit can quickly be very important !

\sphinxAtStartPar
The tasks will be run in \sphinxstylestrong{parallel} on the different CPUs/cores available.

\sphinxAtStartPar
\#\# Documentation and examples :

\sphinxAtStartPar
Here is a basinc example, detailled in a notebook :

\sphinxAtStartPar
{\color{red}\bfseries{}\textasciigrave{}\textasciigrave{}}\textasciigrave{}
import pyterk.config       as config
import pyterk.reporter     as reporter
import pyterk.task\_manager as task\_manager

\sphinxAtStartPar
settings = config.load(‘settings\_example.yml’)

\sphinxAtStartPar
task\_manager.add\_combinational\_iterative\_manyfold(settings, run\_key= ‘Example\sphinxhyphen{}03.1’)
task\_manager.run()

\sphinxAtStartPar
reporter.show\_run\_reports(settings)
{\color{red}\bfseries{}\textasciigrave{}\textasciigrave{}}{\color{red}\bfseries{}\textasciigrave{}}

\sphinxAtStartPar
This will retrieve all settings from \sphinxtitleref{settings\_example.yml}, prepare the different tasks and execute them.
The last call, intended to be used from a Jupyter lab notebook, displays a complete execution report.

\sphinxAtStartPar
You can find \sphinxstylestrong{3 full example notebooks}, with a setting file :
\begin{itemize}
\item {} 
\sphinxAtStartPar
settings\_example.yml

\item {} 
\sphinxAtStartPar
01\sphinxhyphen{}Example\sphinxhyphen{}01.ipynb

\item {} 
\sphinxAtStartPar
02\sphinxhyphen{}Example\sphinxhyphen{}02.ipynb

\item {} 
\sphinxAtStartPar
03\sphinxhyphen{}Example\sphinxhyphen{}03.ipynb

\end{itemize}
\index{VERSION (in module pyterk)@\spxentry{VERSION}\spxextra{in module pyterk}}

\begin{fulllineitems}
\phantomsection\label{\detokenize{pyterk:pyterk.VERSION}}\pysigline{\sphinxcode{\sphinxupquote{pyterk.}}\sphinxbfcode{\sphinxupquote{VERSION}}\sphinxbfcode{\sphinxupquote{\DUrole{w}{  }\DUrole{p}{=}\DUrole{w}{  }2.14}}}
\sphinxAtStartPar
pyterk version

\end{fulllineitems}



\section{Submodules}
\label{\detokenize{pyterk:submodules}}

\section{pyterk.config module}
\label{\detokenize{pyterk:module-pyterk.config}}\label{\detokenize{pyterk:pyterk-config-module}}\index{module@\spxentry{module}!pyterk.config@\spxentry{pyterk.config}}\index{pyterk.config@\spxentry{pyterk.config}!module@\spxentry{module}}
\sphinxAtStartPar
Configuration management.

\sphinxAtStartPar
The settings files allow to specify datasets and models.

\sphinxAtStartPar
\#\# Utilisation:
Loading a settings file :
\sphinxcode{\sphinxupquote{\textasciigrave{}
settings = config.load(\textquotesingle{}settings\_example.yml\textquotesingle{})
\textasciigrave{}}}
or:
{\color{red}\bfseries{}\textasciigrave{}\textasciigrave{}}\textasciigrave{}
settings = config.load(‘settings\_example.yml’,
\begin{quote}

\sphinxAtStartPar
datasets\_dir\_env=’MY\_DATASETS\_DIR’)
\end{quote}

\sphinxAtStartPar
{\color{red}\bfseries{}\textasciigrave{}\textasciigrave{}}{\color{red}\bfseries{}\textasciigrave{}}

\sphinxAtStartPar
where MY\_DATASETS\_DIR is an environment variable that will override \sphinxtitleref{datasets\_dir}
directive in settings file.
\index{datasets (in module pyterk.config)@\spxentry{datasets}\spxextra{in module pyterk.config}}

\begin{fulllineitems}
\phantomsection\label{\detokenize{pyterk:pyterk.config.datasets}}\pysigline{\sphinxcode{\sphinxupquote{pyterk.config.}}\sphinxbfcode{\sphinxupquote{datasets}}\sphinxbfcode{\sphinxupquote{\DUrole{w}{  }\DUrole{p}{=}\DUrole{w}{  }None}}}
\sphinxAtStartPar
datasets profiles

\end{fulllineitems}

\index{load() (in module pyterk.config)@\spxentry{load()}\spxextra{in module pyterk.config}}

\begin{fulllineitems}
\phantomsection\label{\detokenize{pyterk:pyterk.config.load}}\pysiglinewithargsret{\sphinxcode{\sphinxupquote{pyterk.config.}}\sphinxbfcode{\sphinxupquote{load}}}{\emph{\DUrole{n}{filename}}, \emph{\DUrole{n}{datasets\_dir\_env}\DUrole{o}{=}\DUrole{default_value}{\textquotesingle{}PYTERK\_DATASETS\_DIR\textquotesingle{}}}, \emph{\DUrole{n}{run\_dir\_env}\DUrole{o}{=}\DUrole{default_value}{\textquotesingle{}PYTERK\_RUN\_DIR\textquotesingle{}}}, \emph{\DUrole{n}{verbose}\DUrole{o}{=}\DUrole{default_value}{0}}}{}\begin{description}
\item[{Load a setting file and dfined datasets.}] \leavevmode
\sphinxAtStartPar
If given, environment variable can be use to overide \sphinxtitleref{datasets\_dir} directive from setting file.
Usefull for portability between several sites.

\end{description}
\begin{quote}\begin{description}
\item[{Parameters}] \leavevmode\begin{itemize}
\item {} 
\sphinxAtStartPar
\sphinxstyleliteralstrong{\sphinxupquote{filename}} (\sphinxstyleliteralemphasis{\sphinxupquote{string}}) \textendash{} Filename of the yaml setting file

\item {} 
\sphinxAtStartPar
\sphinxstyleliteralstrong{\sphinxupquote{datasets\_dir\_env}} (\sphinxstyleliteralemphasis{\sphinxupquote{string}}) \textendash{} Name of the overiding environment variable

\item {} 
\sphinxAtStartPar
\sphinxstyleliteralstrong{\sphinxupquote{verbose}} (\sphinxstyleliteralemphasis{\sphinxupquote{int}}) \textendash{} verbose mode for loaded datasets (0).

\end{itemize}

\item[{Returns}] \leavevmode
\sphinxAtStartPar
A dict from setting file, completed by datasets and more.

\end{description}\end{quote}

\end{fulllineitems}

\index{models (in module pyterk.config)@\spxentry{models}\spxextra{in module pyterk.config}}

\begin{fulllineitems}
\phantomsection\label{\detokenize{pyterk:pyterk.config.models}}\pysigline{\sphinxcode{\sphinxupquote{pyterk.config.}}\sphinxbfcode{\sphinxupquote{models}}\sphinxbfcode{\sphinxupquote{\DUrole{w}{  }\DUrole{p}{=}\DUrole{w}{  }None}}}
\sphinxAtStartPar
models profiles

\end{fulllineitems}

\index{run\_dir (in module pyterk.config)@\spxentry{run\_dir}\spxextra{in module pyterk.config}}

\begin{fulllineitems}
\phantomsection\label{\detokenize{pyterk:pyterk.config.run_dir}}\pysigline{\sphinxcode{\sphinxupquote{pyterk.config.}}\sphinxbfcode{\sphinxupquote{run\_dir}}\sphinxbfcode{\sphinxupquote{\DUrole{w}{  }\DUrole{p}{=}\DUrole{w}{  }None}}}
\sphinxAtStartPar
run\_dir, the place to put all output directories

\end{fulllineitems}

\index{runs (in module pyterk.config)@\spxentry{runs}\spxextra{in module pyterk.config}}

\begin{fulllineitems}
\phantomsection\label{\detokenize{pyterk:pyterk.config.runs}}\pysigline{\sphinxcode{\sphinxupquote{pyterk.config.}}\sphinxbfcode{\sphinxupquote{runs}}\sphinxbfcode{\sphinxupquote{\DUrole{w}{  }\DUrole{p}{=}\DUrole{w}{  }None}}}
\sphinxAtStartPar
dict of runs section

\end{fulllineitems}

\index{settings (in module pyterk.config)@\spxentry{settings}\spxextra{in module pyterk.config}}

\begin{fulllineitems}
\phantomsection\label{\detokenize{pyterk:pyterk.config.settings}}\pysigline{\sphinxcode{\sphinxupquote{pyterk.config.}}\sphinxbfcode{\sphinxupquote{settings}}\sphinxbfcode{\sphinxupquote{\DUrole{w}{  }\DUrole{p}{=}\DUrole{w}{  }None}}}
\sphinxAtStartPar
Dict of settings

\end{fulllineitems}



\section{pyterk.models module}
\label{\detokenize{pyterk:module-pyterk.models}}\label{\detokenize{pyterk:pyterk-models-module}}\index{module@\spxentry{module}!pyterk.models@\spxentry{pyterk.models}}\index{pyterk.models@\spxentry{pyterk.models}!module@\spxentry{module}}
\sphinxAtStartPar
This module is for internal use only \sphinxhyphen{} You do not have to interact with ;\sphinxhyphen{}).
\index{get\_model() (in module pyterk.models)@\spxentry{get\_model()}\spxextra{in module pyterk.models}}

\begin{fulllineitems}
\phantomsection\label{\detokenize{pyterk:pyterk.models.get_model}}\pysiglinewithargsret{\sphinxcode{\sphinxupquote{pyterk.models.}}\sphinxbfcode{\sphinxupquote{get\_model}}}{\emph{\DUrole{n}{profile}}}{}
\sphinxAtStartPar
Get a model from a model profile.
The profile contains the module and function name of the model, and the arguments.
The model will be retrieved by calling the function with the arguments.
:param profile: a model profile
:type profile: dict
\begin{quote}\begin{description}
\item[{Returns}] \leavevmode
\sphinxAtStartPar
keras model as defined in the profile.

\item[{Return type}] \leavevmode
\sphinxAtStartPar
model (keras model)

\end{description}\end{quote}

\end{fulllineitems}



\section{pyterk.reporter module}
\label{\detokenize{pyterk:module-pyterk.reporter}}\label{\detokenize{pyterk:pyterk-reporter-module}}\index{module@\spxentry{module}!pyterk.reporter@\spxentry{pyterk.reporter}}\index{pyterk.reporter@\spxentry{pyterk.reporter}!module@\spxentry{module}}
\sphinxAtStartPar
Module to generate execution reports.

\sphinxAtStartPar
During the run of the tasks, the bestmodel and results are saved in h5 and json files:
\begin{itemize}
\item {} 
\sphinxAtStartPar
\sphinxtitleref{about.json} : information and description of the task

\item {} 
\sphinxAtStartPar
\sphinxtitleref{history.json} : history from model.fit()

\item {} 
\sphinxAtStartPar
\sphinxtitleref{evaluation.json} : evaluation from model.evaluate()

\item {} 
\sphinxAtStartPar
\sphinxtitleref{bestmodel.h5} : best model

\end{itemize}

\sphinxAtStartPar
\#\#\# Example :

\sphinxAtStartPar
{\color{red}\bfseries{}\textasciigrave{}\textasciigrave{}}\textasciigrave{}
reporter.show\_run\_reports(settings,
\begin{quote}

\sphinxAtStartPar
args   = {[}‘dataset\_id’,’model\_id’,’batch\_size’{]},
evaluation = {[}2{]})
\end{quote}

\sphinxAtStartPar
{\color{red}\bfseries{}\textasciigrave{}\textasciigrave{}}{\color{red}\bfseries{}\textasciigrave{}}

\sphinxAtStartPar
This module will retrieve information from json files and generate a report.
\index{plot\_confusion() (in module pyterk.reporter)@\spxentry{plot\_confusion()}\spxextra{in module pyterk.reporter}}

\begin{fulllineitems}
\phantomsection\label{\detokenize{pyterk:pyterk.reporter.plot_confusion}}\pysiglinewithargsret{\sphinxcode{\sphinxupquote{pyterk.reporter.}}\sphinxbfcode{\sphinxupquote{plot\_confusion}}}{\emph{\DUrole{n}{run\_dir}}, \emph{\DUrole{n}{predict\_type}\DUrole{o}{=}\DUrole{default_value}{\textquotesingle{}softmax\textquotesingle{}}}, \emph{\DUrole{n}{normalize}\DUrole{o}{=}\DUrole{default_value}{\textquotesingle{}pred\textquotesingle{}}}, \emph{\DUrole{n}{figsize}\DUrole{o}{=}\DUrole{default_value}{(5, 5)}}, \emph{\DUrole{n}{savefig}\DUrole{o}{=}\DUrole{default_value}{True}}, \emph{\DUrole{n}{mplstyle}\DUrole{o}{=}\DUrole{default_value}{\textquotesingle{}pyterk\textquotesingle{}}}}{}
\sphinxAtStartPar
Plot a confusion matrix
\begin{quote}\begin{description}
\item[{Parameters}] \leavevmode\begin{itemize}
\item {} 
\sphinxAtStartPar
\sphinxstyleliteralstrong{\sphinxupquote{iterations\_dir}} \textendash{} a directory with iterations subdirs (iter\sphinxhyphen{}000, iter\sphinxhyphen{}001, …)

\item {} 
\sphinxAtStartPar
\sphinxstyleliteralstrong{\sphinxupquote{predict\_type}} \textendash{} sigmoid, softmax or classes

\item {} 
\sphinxAtStartPar
\sphinxstyleliteralstrong{\sphinxupquote{normalize}} \textendash{} true, pred, all or None (pred)

\item {} 
\sphinxAtStartPar
\sphinxstyleliteralstrong{\sphinxupquote{figsize}} \textendash{} figure size

\item {} 
\sphinxAtStartPar
\sphinxstyleliteralstrong{\sphinxupquote{savefig}} \textendash{} save fig (True) or not (False)

\end{itemize}

\item[{Returns}] \leavevmode
\sphinxAtStartPar
Just plot the matrix and print report and hamming loss

\end{description}\end{quote}

\end{fulllineitems}

\index{plot\_distribution() (in module pyterk.reporter)@\spxentry{plot\_distribution()}\spxextra{in module pyterk.reporter}}

\begin{fulllineitems}
\phantomsection\label{\detokenize{pyterk:pyterk.reporter.plot_distribution}}\pysiglinewithargsret{\sphinxcode{\sphinxupquote{pyterk.reporter.}}\sphinxbfcode{\sphinxupquote{plot\_distribution}}}{\emph{\DUrole{n}{run\_dir}}, \emph{\DUrole{n}{metric\_id}\DUrole{o}{=}\DUrole{default_value}{0}}, \emph{\DUrole{n}{bins}\DUrole{o}{=}\DUrole{default_value}{10}}, \emph{\DUrole{n}{min}\DUrole{o}{=}\DUrole{default_value}{None}}, \emph{\DUrole{n}{max}\DUrole{o}{=}\DUrole{default_value}{None}}, \emph{\DUrole{n}{figsize}\DUrole{o}{=}\DUrole{default_value}{(10, 8)}}, \emph{\DUrole{n}{savefig}\DUrole{o}{=}\DUrole{default_value}{False}}, \emph{\DUrole{n}{mplstyle}\DUrole{o}{=}\DUrole{default_value}{\textquotesingle{}pyterk\textquotesingle{}}}}{}
\sphinxAtStartPar
Plot distribution of a given metric from an evaluation.json saved file.
For a kfold or an iterative kfold, all evaluation data are concatened in an evaluation.json file in main run\_dir.
\begin{quote}\begin{description}
\item[{Parameters}] \leavevmode\begin{itemize}
\item {} 
\sphinxAtStartPar
\sphinxstyleliteralstrong{\sphinxupquote{run\_dir}} (\sphinxstyleliteralemphasis{\sphinxupquote{string}}) \textendash{} directory path of json evaluation file

\item {} 
\sphinxAtStartPar
\sphinxstyleliteralstrong{\sphinxupquote{metricid}} (\sphinxstyleliteralemphasis{\sphinxupquote{int}}) \textendash{} number of metric to plot. Example : 2

\item {} 
\sphinxAtStartPar
\sphinxstyleliteralstrong{\sphinxupquote{min}} (\sphinxstyleliteralemphasis{\sphinxupquote{int}}) \textendash{} min value

\item {} 
\sphinxAtStartPar
\sphinxstyleliteralstrong{\sphinxupquote{max}} (\sphinxstyleliteralemphasis{\sphinxupquote{int}}) \textendash{} max value

\item {} 
\sphinxAtStartPar
\sphinxstyleliteralstrong{\sphinxupquote{bins}} (\sphinxstyleliteralemphasis{\sphinxupquote{int}}) \textendash{} number of bins

\item {} 
\sphinxAtStartPar
\sphinxstyleliteralstrong{\sphinxupquote{figsize}} (\sphinxstyleliteralemphasis{\sphinxupquote{tuple}}) \textendash{} figure size, default is (10,8)

\item {} 
\sphinxAtStartPar
\sphinxstyleliteralstrong{\sphinxupquote{savefig}} (\sphinxstyleliteralemphasis{\sphinxupquote{boolean}}) \textendash{} if True, figure will be save in run\_dir.

\item {} 
\sphinxAtStartPar
\sphinxstyleliteralstrong{\sphinxupquote{mplstyle}} (\sphinxstyleliteralemphasis{\sphinxupquote{string}}) \textendash{} name of matplotlib style. default is ‘pyterk’, but all matplotlib are ok (default, bmh, …)

\end{itemize}

\item[{Returns}] \leavevmode
\sphinxAtStartPar
Nothing, but display a beautifull distribution plot !

\end{description}\end{quote}

\end{fulllineitems}

\index{plot\_history() (in module pyterk.reporter)@\spxentry{plot\_history()}\spxextra{in module pyterk.reporter}}

\begin{fulllineitems}
\phantomsection\label{\detokenize{pyterk:pyterk.reporter.plot_history}}\pysiglinewithargsret{\sphinxcode{\sphinxupquote{pyterk.reporter.}}\sphinxbfcode{\sphinxupquote{plot\_history}}}{\emph{\DUrole{n}{run\_dir}}, \emph{\DUrole{n}{metric}\DUrole{o}{=}\DUrole{default_value}{\textquotesingle{}val\_mae\textquotesingle{}}}, \emph{\DUrole{n}{min}\DUrole{o}{=}\DUrole{default_value}{None}}, \emph{\DUrole{n}{max}\DUrole{o}{=}\DUrole{default_value}{None}}, \emph{\DUrole{n}{figsize}\DUrole{o}{=}\DUrole{default_value}{(10, 8)}}, \emph{\DUrole{n}{savefig}\DUrole{o}{=}\DUrole{default_value}{False}}, \emph{\DUrole{n}{mplstyle}\DUrole{o}{=}\DUrole{default_value}{\textquotesingle{}pyterk\textquotesingle{}}}}{}
\sphinxAtStartPar
Plot history evolution from history.json saved file.
For a kfold or an iterative kfold, all history data are concatened in history.json file in main run\_dir.
This will plot a curve for each one in a common plot.
\begin{quote}\begin{description}
\item[{Parameters}] \leavevmode\begin{itemize}
\item {} 
\sphinxAtStartPar
\sphinxstyleliteralstrong{\sphinxupquote{run\_dir}} (\sphinxstyleliteralemphasis{\sphinxupquote{string}}) \textendash{} directory path of json history file

\item {} 
\sphinxAtStartPar
\sphinxstyleliteralstrong{\sphinxupquote{metric}} (\sphinxstyleliteralemphasis{\sphinxupquote{string}}) \textendash{} metric name to plot. Example : ‘val\_mae’

\item {} 
\sphinxAtStartPar
\sphinxstyleliteralstrong{\sphinxupquote{figsize}} (\sphinxstyleliteralemphasis{\sphinxupquote{tuple}}) \textendash{} figure size, default is (10,8)

\item {} 
\sphinxAtStartPar
\sphinxstyleliteralstrong{\sphinxupquote{savefig}} (\sphinxstyleliteralemphasis{\sphinxupquote{boolean}}) \textendash{} if True, figure will be save in run\_dir.

\item {} 
\sphinxAtStartPar
\sphinxstyleliteralstrong{\sphinxupquote{mplstyle}} (\sphinxstyleliteralemphasis{\sphinxupquote{string}}) \textendash{} name of matplotlib style. default is ‘pyterk’, but all matplotlib are ok (default, bmh, …)

\end{itemize}

\item[{Returns}] \leavevmode
\sphinxAtStartPar
Nothing, but display a beautifull plot !

\end{description}\end{quote}

\end{fulllineitems}

\index{plot\_kfold\_correlation() (in module pyterk.reporter)@\spxentry{plot\_kfold\_correlation()}\spxextra{in module pyterk.reporter}}

\begin{fulllineitems}
\phantomsection\label{\detokenize{pyterk:pyterk.reporter.plot_kfold_correlation}}\pysiglinewithargsret{\sphinxcode{\sphinxupquote{pyterk.reporter.}}\sphinxbfcode{\sphinxupquote{plot\_kfold\_correlation}}}{\emph{\DUrole{n}{run\_dir}}, \emph{\DUrole{n}{channel}\DUrole{o}{=}\DUrole{default_value}{0}}, \emph{\DUrole{n}{figsize}\DUrole{o}{=}\DUrole{default_value}{(8, 6)}}, \emph{\DUrole{n}{axes\_min}\DUrole{o}{=}\DUrole{default_value}{\textquotesingle{}auto\textquotesingle{}}}, \emph{\DUrole{n}{axes\_max}\DUrole{o}{=}\DUrole{default_value}{\textquotesingle{}auto\textquotesingle{}}}, \emph{\DUrole{n}{yy\_deltamax}\DUrole{o}{=}\DUrole{default_value}{None}}, \emph{\DUrole{n}{marker}\DUrole{o}{=}\DUrole{default_value}{\textquotesingle{}o\textquotesingle{}}}, \emph{\DUrole{n}{markersize}\DUrole{o}{=}\DUrole{default_value}{8}}, \emph{\DUrole{n}{alpha}\DUrole{o}{=}\DUrole{default_value}{0.7}}, \emph{\DUrole{n}{color}\DUrole{o}{=}\DUrole{default_value}{\textquotesingle{}auto\textquotesingle{}}}, \emph{\DUrole{n}{savefig}\DUrole{o}{=}\DUrole{default_value}{True}}, \emph{\DUrole{n}{mplstyle}\DUrole{o}{=}\DUrole{default_value}{\textquotesingle{}pyterk\textquotesingle{}}}}{}
\sphinxAtStartPar
Plot a correlation for a (y\_test, y\_pred) saved json file.
\begin{quote}\begin{description}
\item[{Parameters}] \leavevmode\begin{itemize}
\item {} 
\sphinxAtStartPar
\sphinxstyleliteralstrong{\sphinxupquote{run\_file}} \textendash{} a manyfold directory where kfold subdirectories are

\item {} 
\sphinxAtStartPar
\sphinxstyleliteralstrong{\sphinxupquote{channel}} \textendash{} composant of y to plot

\item {} 
\sphinxAtStartPar
\sphinxstyleliteralstrong{\sphinxupquote{figsize}} (\sphinxstyleliteralemphasis{\sphinxupquote{tuple}}) \textendash{} figure size, default is (10,8)

\item {} 
\sphinxAtStartPar
\sphinxstyleliteralstrong{\sphinxupquote{axes\_min}} \textendash{} min value for x and y axe. ‘auto’ or float

\item {} 
\sphinxAtStartPar
\sphinxstyleliteralstrong{\sphinxupquote{axes\_max}} \textendash{} max value for x and y axe. ‘auto’ or float

\item {} 
\sphinxAtStartPar
\sphinxstyleliteralstrong{\sphinxupquote{mplstyle}} (\sphinxstyleliteralemphasis{\sphinxupquote{string}}) \textendash{} name of matplotlib style. default is ‘pyterk’, but all matplotlib are ok (default, bmh, …)

\item {} 
\sphinxAtStartPar
\sphinxstyleliteralstrong{\sphinxupquote{marker}} \textendash{} marker, default is ‘.’

\item {} 
\sphinxAtStartPar
\sphinxstyleliteralstrong{\sphinxupquote{markersize}} \textendash{} marker size

\item {} 
\sphinxAtStartPar
\sphinxstyleliteralstrong{\sphinxupquote{alpha}} \textendash{} marker alpha

\item {} 
\sphinxAtStartPar
\sphinxstyleliteralstrong{\sphinxupquote{color}} \textendash{} plot color or ‘auto’

\item {} 
\sphinxAtStartPar
\sphinxstyleliteralstrong{\sphinxupquote{savefig}} \textendash{} if True, save fig in run\_dir

\end{itemize}

\item[{Returns}] \leavevmode
\sphinxAtStartPar
\sphinxhyphen{})

\item[{Return type}] \leavevmode
\sphinxAtStartPar
Nothing, but display a beautifull correlation plot

\end{description}\end{quote}

\end{fulllineitems}

\index{show\_report() (in module pyterk.reporter)@\spxentry{show\_report()}\spxextra{in module pyterk.reporter}}

\begin{fulllineitems}
\phantomsection\label{\detokenize{pyterk:pyterk.reporter.show_report}}\pysiglinewithargsret{\sphinxcode{\sphinxupquote{pyterk.reporter.}}\sphinxbfcode{\sphinxupquote{show\_report}}}{\emph{\DUrole{n}{run\_dir}}, \emph{\DUrole{n}{padding}\DUrole{o}{=}\DUrole{default_value}{\textquotesingle{}\textquotesingle{}}}, \emph{\DUrole{n}{sections}\DUrole{o}{=}\DUrole{default_value}{{[}\textquotesingle{}title\textquotesingle{}, \textquotesingle{}context\textquotesingle{}, \textquotesingle{}args\textquotesingle{}, \textquotesingle{}settings\textquotesingle{}, \textquotesingle{}evaluation\textquotesingle{}, \textquotesingle{}monitoring\textquotesingle{}, \textquotesingle{}history\textquotesingle{}, \textquotesingle{}distribution\textquotesingle{}, \textquotesingle{}correlation\textquotesingle{}{]}}}, \emph{\DUrole{n}{context}\DUrole{o}{=}\DUrole{default_value}{{[}\textquotesingle{}function\textquotesingle{}, \textquotesingle{}version\textquotesingle{}, \textquotesingle{}date\textquotesingle{}, \textquotesingle{}description\textquotesingle{}, \textquotesingle{}seed\textquotesingle{}{]}}}, \emph{\DUrole{n}{args}\DUrole{o}{=}\DUrole{default_value}{{[}\textquotesingle{}run\_dir\textquotesingle{}, \textquotesingle{}dataset\_id\textquotesingle{}, \textquotesingle{}model\_id\textquotesingle{}, \textquotesingle{}n\_iter\textquotesingle{}, \textquotesingle{}k\_fold\textquotesingle{}, \textquotesingle{}epochs\textquotesingle{}, \textquotesingle{}batch\_size\textquotesingle{}{]}}}, \emph{\DUrole{n}{settings}\DUrole{o}{=}\DUrole{default_value}{{[}\textquotesingle{}file\textquotesingle{}, \textquotesingle{}version\textquotesingle{}, \textquotesingle{}description\textquotesingle{}, \textquotesingle{}datasets\_dir\textquotesingle{}, \textquotesingle{}run\_dir\textquotesingle{}{]}}}, \emph{\DUrole{n}{evaluation}\DUrole{o}{=}\DUrole{default_value}{{[}\textquotesingle{}all\textquotesingle{}{]}}}, \emph{\DUrole{n}{monitoring}\DUrole{o}{=}\DUrole{default_value}{{[}\textquotesingle{}duration\textquotesingle{}, \textquotesingle{}used\_data\textquotesingle{}{]}}}, \emph{\DUrole{n}{history}\DUrole{o}{=}\DUrole{default_value}{{[}\{\textquotesingle{}metric\textquotesingle{}: \textquotesingle{}val\_mae\textquotesingle{}, \textquotesingle{}min\textquotesingle{}: None, \textquotesingle{}max\textquotesingle{}: None, \textquotesingle{}figsize\textquotesingle{}: (8, 6), \textquotesingle{}savefig\textquotesingle{}: True, \textquotesingle{}mplstyle\textquotesingle{}: \textquotesingle{}pyterk\textquotesingle{}\}{]}}}, \emph{\DUrole{n}{distribution}\DUrole{o}{=}\DUrole{default_value}{{[}\{\textquotesingle{}metric\_id\textquotesingle{}: 2, \textquotesingle{}bins\textquotesingle{}: 4, \textquotesingle{}min\textquotesingle{}: None, \textquotesingle{}max\textquotesingle{}: None, \textquotesingle{}figsize\textquotesingle{}: (8, 6), \textquotesingle{}savefig\textquotesingle{}: True, \textquotesingle{}mplstyle\textquotesingle{}: \textquotesingle{}pyterk\textquotesingle{}\}{]}}}, \emph{\DUrole{n}{correlation}\DUrole{o}{=}\DUrole{default_value}{{[}\{\textquotesingle{}axes\_min\textquotesingle{}: \textquotesingle{}auto\textquotesingle{}, \textquotesingle{}axes\_max\textquotesingle{}: \textquotesingle{}auto\textquotesingle{}, \textquotesingle{}figsize\textquotesingle{}: (8, 6), \textquotesingle{}marker\textquotesingle{}: \textquotesingle{}.\textquotesingle{}, \textquotesingle{}markersize\textquotesingle{}: 8, \textquotesingle{}alpha\textquotesingle{}: 0.7, \textquotesingle{}color\textquotesingle{}: \textquotesingle{}auto\textquotesingle{}, \textquotesingle{}savefig\textquotesingle{}: True, \textquotesingle{}mplstyle\textquotesingle{}: \textquotesingle{}pyterk\textquotesingle{}\}{]}}}, \emph{\DUrole{n}{confusion}\DUrole{o}{=}\DUrole{default_value}{{[}\{\textquotesingle{}normalize\textquotesingle{}: \textquotesingle{}pred\textquotesingle{}, \textquotesingle{}predict\_type\textquotesingle{}: \textquotesingle{}softmax\textquotesingle{}, \textquotesingle{}figsize\textquotesingle{}: (5, 5), \textquotesingle{}savefig\textquotesingle{}: True, \textquotesingle{}mplstyle\textquotesingle{}: \textquotesingle{}pyterk\textquotesingle{}\}{]}}}}{}
\sphinxAtStartPar
Builds and displays a report from the json data of a given run\_dir.
\begin{quote}\begin{description}
\item[{Parameters}] \leavevmode\begin{itemize}
\item {} 
\sphinxAtStartPar
\sphinxstyleliteralstrong{\sphinxupquote{run\_dir}} (\sphinxstyleliteralemphasis{\sphinxupquote{string}}) \textendash{} directory path of json report file

\item {} 
\sphinxAtStartPar
\sphinxstyleliteralstrong{\sphinxupquote{sections}} (\sphinxstyleliteralemphasis{\sphinxupquote{list}}) \textendash{} list of sections to include in the report

\item {} 
\sphinxAtStartPar
\sphinxstyleliteralstrong{\sphinxupquote{context}} (\sphinxstyleliteralemphasis{\sphinxupquote{list}}) \textendash{} informations to include in context section

\item {} 
\sphinxAtStartPar
\sphinxstyleliteralstrong{\sphinxupquote{args}} (\sphinxstyleliteralemphasis{\sphinxupquote{list}}) \textendash{} informations to include in args section

\item {} 
\sphinxAtStartPar
\sphinxstyleliteralstrong{\sphinxupquote{settings}} (\sphinxstyleliteralemphasis{\sphinxupquote{list}}) \textendash{} informations to include in settings section

\item {} 
\sphinxAtStartPar
\sphinxstyleliteralstrong{\sphinxupquote{evaluation}} (\sphinxstyleliteralemphasis{\sphinxupquote{list}}) \textendash{} \#metrics to include in evaluation section. ‘all’ mean all. Example : {[}0,1,2{]}

\item {} 
\sphinxAtStartPar
\sphinxstyleliteralstrong{\sphinxupquote{history}} (\sphinxstyleliteralemphasis{\sphinxupquote{dict}}) \textendash{} parameters for history plot \sphinxhyphen{} see \sphinxtitleref{plot\_history}

\item {} 
\sphinxAtStartPar
\sphinxstyleliteralstrong{\sphinxupquote{distribution}} (\sphinxstyleliteralemphasis{\sphinxupquote{dict}}) \textendash{} parameters for metrics distribution plot

\item {} 
\sphinxAtStartPar
\sphinxstyleliteralstrong{\sphinxupquote{correlation}} (\sphinxstyleliteralemphasis{\sphinxupquote{dict}}) \textendash{} parameters for correlation plot

\item {} 
\sphinxAtStartPar
\sphinxstyleliteralstrong{\sphinxupquote{confusion}} (\sphinxstyleliteralemphasis{\sphinxupquote{dict}}) \textendash{} parameters for confusion matrix (need yytest files)

\end{itemize}

\end{description}\end{quote}

\end{fulllineitems}

\index{show\_run\_reports() (in module pyterk.reporter)@\spxentry{show\_run\_reports()}\spxextra{in module pyterk.reporter}}

\begin{fulllineitems}
\phantomsection\label{\detokenize{pyterk:pyterk.reporter.show_run_reports}}\pysiglinewithargsret{\sphinxcode{\sphinxupquote{pyterk.reporter.}}\sphinxbfcode{\sphinxupquote{show\_run\_reports}}}{\emph{\DUrole{n}{run\_config}}, \emph{\DUrole{n}{run\_filter}\DUrole{o}{=}\DUrole{default_value}{\textquotesingle{}.*\textquotesingle{}}}, \emph{\DUrole{n}{sections}\DUrole{o}{=}\DUrole{default_value}{{[}\textquotesingle{}title\textquotesingle{}, \textquotesingle{}context\textquotesingle{}, \textquotesingle{}args\textquotesingle{}, \textquotesingle{}settings\textquotesingle{}, \textquotesingle{}evaluation\textquotesingle{}, \textquotesingle{}monitoring\textquotesingle{}, \textquotesingle{}history\textquotesingle{}, \textquotesingle{}distribution\textquotesingle{}, \textquotesingle{}correlation\textquotesingle{}, \textquotesingle{}confusion\textquotesingle{}{]}}}, \emph{\DUrole{n}{context}\DUrole{o}{=}\DUrole{default_value}{{[}\textquotesingle{}function\textquotesingle{}, \textquotesingle{}version\textquotesingle{}, \textquotesingle{}date\textquotesingle{}, \textquotesingle{}description\textquotesingle{}, \textquotesingle{}seed\textquotesingle{}{]}}}, \emph{\DUrole{n}{args}\DUrole{o}{=}\DUrole{default_value}{{[}\textquotesingle{}run\_dir\textquotesingle{}, \textquotesingle{}dataset\_id\textquotesingle{}, \textquotesingle{}model\_id\textquotesingle{}, \textquotesingle{}n\_iter\textquotesingle{}, \textquotesingle{}k\_fold\textquotesingle{}, \textquotesingle{}epochs\textquotesingle{}, \textquotesingle{}batch\_size\textquotesingle{}{]}}}, \emph{\DUrole{n}{settings}\DUrole{o}{=}\DUrole{default_value}{{[}\textquotesingle{}file\textquotesingle{}, \textquotesingle{}version\textquotesingle{}, \textquotesingle{}description\textquotesingle{}, \textquotesingle{}datasets\_dir\textquotesingle{}, \textquotesingle{}run\_dir\textquotesingle{}{]}}}, \emph{\DUrole{n}{evaluation}\DUrole{o}{=}\DUrole{default_value}{{[}\textquotesingle{}all\textquotesingle{}{]}}}, \emph{\DUrole{n}{monitoring}\DUrole{o}{=}\DUrole{default_value}{{[}\textquotesingle{}duration\textquotesingle{}, \textquotesingle{}used\_data\textquotesingle{}{]}}}, \emph{\DUrole{n}{history}\DUrole{o}{=}\DUrole{default_value}{{[}\{\textquotesingle{}metric\textquotesingle{}: \textquotesingle{}val\_mae\textquotesingle{}, \textquotesingle{}min\textquotesingle{}: None, \textquotesingle{}max\textquotesingle{}: None, \textquotesingle{}figsize\textquotesingle{}: (8, 6), \textquotesingle{}savefig\textquotesingle{}: True, \textquotesingle{}mplstyle\textquotesingle{}: \textquotesingle{}pyterk\textquotesingle{}\}{]}}}, \emph{\DUrole{n}{distribution}\DUrole{o}{=}\DUrole{default_value}{{[}\{\textquotesingle{}metric\_id\textquotesingle{}: 2, \textquotesingle{}bins\textquotesingle{}: 4, \textquotesingle{}min\textquotesingle{}: None, \textquotesingle{}max\textquotesingle{}: None, \textquotesingle{}figsize\textquotesingle{}: (8, 6), \textquotesingle{}savefig\textquotesingle{}: True, \textquotesingle{}mplstyle\textquotesingle{}: \textquotesingle{}pyterk\textquotesingle{}\}{]}}}, \emph{\DUrole{n}{correlation}\DUrole{o}{=}\DUrole{default_value}{{[}\{\textquotesingle{}axes\_min\textquotesingle{}: \textquotesingle{}auto\textquotesingle{}, \textquotesingle{}axes\_max\textquotesingle{}: \textquotesingle{}auto\textquotesingle{}, \textquotesingle{}figsize\textquotesingle{}: (8, 6), \textquotesingle{}marker\textquotesingle{}: \textquotesingle{}.\textquotesingle{}, \textquotesingle{}markersize\textquotesingle{}: 8, \textquotesingle{}alpha\textquotesingle{}: 0.7, \textquotesingle{}color\textquotesingle{}: \textquotesingle{}auto\textquotesingle{}, \textquotesingle{}savefig\textquotesingle{}: True, \textquotesingle{}mplstyle\textquotesingle{}: \textquotesingle{}pyterk\textquotesingle{}\}{]}}}, \emph{\DUrole{n}{confusion}\DUrole{o}{=}\DUrole{default_value}{{[}\{\textquotesingle{}normalize\textquotesingle{}: \textquotesingle{}pred\textquotesingle{}, \textquotesingle{}predict\_type\textquotesingle{}: \textquotesingle{}softmax\textquotesingle{}, \textquotesingle{}figsize\textquotesingle{}: (5, 5), \textquotesingle{}savefig\textquotesingle{}: True, \textquotesingle{}mplstyle\textquotesingle{}: \textquotesingle{}pyterk\textquotesingle{}\}{]}}}}{}
\sphinxAtStartPar
Displays a full report in two parts, short and long, for all runs defined in the settings.
Very simple to use…
\begin{quote}\begin{description}
\item[{Parameters}] \leavevmode\begin{itemize}
\item {} 
\sphinxAtStartPar
\sphinxstyleliteralstrong{\sphinxupquote{run\_config}} (\sphinxstyleliteralemphasis{\sphinxupquote{dict}}) \textendash{} settings, issued from config.load()

\item {} 
\sphinxAtStartPar
\sphinxstyleliteralstrong{\sphinxupquote{run\_filter}} (\sphinxstyleliteralemphasis{\sphinxupquote{regx}}) \textendash{} regex to filter run entries from yml settings file (.*)

\item {} 
\sphinxAtStartPar
\sphinxstyleliteralstrong{\sphinxupquote{sections}} (\sphinxstyleliteralemphasis{\sphinxupquote{list}}) \textendash{} list of sections to include in the report

\item {} 
\sphinxAtStartPar
\sphinxstyleliteralstrong{\sphinxupquote{context}} (\sphinxstyleliteralemphasis{\sphinxupquote{list}}) \textendash{} informations to include in context section

\item {} 
\sphinxAtStartPar
\sphinxstyleliteralstrong{\sphinxupquote{args}} (\sphinxstyleliteralemphasis{\sphinxupquote{list}}) \textendash{} informations to include in args section

\item {} 
\sphinxAtStartPar
\sphinxstyleliteralstrong{\sphinxupquote{settings}} (\sphinxstyleliteralemphasis{\sphinxupquote{list}}) \textendash{} informations to include in settings section

\item {} 
\sphinxAtStartPar
\sphinxstyleliteralstrong{\sphinxupquote{evaluation}} (\sphinxstyleliteralemphasis{\sphinxupquote{list}}) \textendash{} \#metrics to include in evaluation section. ‘all’ mean all. Example : {[}0,1,2{]}

\item {} 
\sphinxAtStartPar
\sphinxstyleliteralstrong{\sphinxupquote{history}} (\sphinxstyleliteralemphasis{\sphinxupquote{dict}}) \textendash{} parameters for history plot \sphinxhyphen{} see \sphinxtitleref{plot\_history}

\item {} 
\sphinxAtStartPar
\sphinxstyleliteralstrong{\sphinxupquote{distribution}} (\sphinxstyleliteralemphasis{\sphinxupquote{dict}}) \textendash{} parameters for metrics distribution plot

\item {} 
\sphinxAtStartPar
\sphinxstyleliteralstrong{\sphinxupquote{correlation}} (\sphinxstyleliteralemphasis{\sphinxupquote{dict}}) \textendash{} parameters for correlation plot

\item {} 
\sphinxAtStartPar
\sphinxstyleliteralstrong{\sphinxupquote{confusion}} (\sphinxstyleliteralemphasis{\sphinxupquote{dict}}) \textendash{} parameters for confusion matrix (need yytest files)

\end{itemize}

\item[{Returns}] \leavevmode
\sphinxAtStartPar
Nothing, but display a short and long report, with index.

\end{description}\end{quote}

\end{fulllineitems}



\section{pyterk.task\_manager module}
\label{\detokenize{pyterk:module-pyterk.task_manager}}\label{\detokenize{pyterk:pyterk-task-manager-module}}\index{module@\spxentry{module}!pyterk.task\_manager@\spxentry{pyterk.task\_manager}}\index{pyterk.task\_manager@\spxentry{pyterk.task\_manager}!module@\spxentry{module}}
\sphinxAtStartPar
Allows to generate tasks and to execute them in a distributed way.

\sphinxAtStartPar
See example notebook : \sphinxtitleref{03\sphinxhyphen{}Example\sphinxhyphen{}03.ipynb}

\sphinxAtStartPar
Example :
{\color{red}\bfseries{}\textasciigrave{}\textasciigrave{}}\textasciigrave{}
task\_manager.add\_combinational\_iterative\_manyfold(settings = settings,
\begin{quote}

\sphinxAtStartPar
run\_key= ‘Example\sphinxhyphen{}03.3’)
\end{quote}

\sphinxAtStartPar
{\color{red}\bfseries{}\textasciigrave{}\textasciigrave{}}{\color{red}\bfseries{}\textasciigrave{}}
\index{add\_combinational\_iterative\_manyfold() (in module pyterk.task\_manager)@\spxentry{add\_combinational\_iterative\_manyfold()}\spxextra{in module pyterk.task\_manager}}

\begin{fulllineitems}
\phantomsection\label{\detokenize{pyterk:pyterk.task_manager.add_combinational_iterative_manyfold}}\pysiglinewithargsret{\sphinxcode{\sphinxupquote{pyterk.task\_manager.}}\sphinxbfcode{\sphinxupquote{add\_combinational\_iterative\_manyfold}}}{\emph{\DUrole{n}{settings}\DUrole{o}{=}\DUrole{default_value}{None}}, \emph{\DUrole{n}{run\_key}\DUrole{o}{=}\DUrole{default_value}{None}}, \emph{\DUrole{n}{verbose}\DUrole{o}{=}\DUrole{default_value}{1}}}{}
\sphinxAtStartPar
Add tasks for a combinational iterative manyfold \sphinxhyphen{} \sphinxtitleref{see 03\sphinxhyphen{}Example\sphinxhyphen{}03}.ipynb
Generates all the tasks of the combinatorial described in the run section of the settings file.
:param settings: settings
:type settings: dict
:param run\_key: name of the config run section
:type run\_key: string
:param verbose: verbosity of generated tasks
:type verbose: int
\begin{quote}\begin{description}
\item[{Returns}] \leavevmode
\sphinxAtStartPar
Nothings. Task are added to the pending taks queue.

\end{description}\end{quote}

\end{fulllineitems}

\index{add\_iterative\_manyfold() (in module pyterk.task\_manager)@\spxentry{add\_iterative\_manyfold()}\spxextra{in module pyterk.task\_manager}}

\begin{fulllineitems}
\phantomsection\label{\detokenize{pyterk:pyterk.task_manager.add_iterative_manyfold}}\pysiglinewithargsret{\sphinxcode{\sphinxupquote{pyterk.task\_manager.}}\sphinxbfcode{\sphinxupquote{add\_iterative\_manyfold}}}{\emph{\DUrole{n}{settings}\DUrole{o}{=}\DUrole{default_value}{None}}, \emph{\DUrole{n}{run\_dir}\DUrole{o}{=}\DUrole{default_value}{None}}, \emph{\DUrole{n}{dataset\_id}\DUrole{o}{=}\DUrole{default_value}{None}}, \emph{\DUrole{n}{model\_id}\DUrole{o}{=}\DUrole{default_value}{None}}, \emph{\DUrole{n}{n\_iter}\DUrole{o}{=}\DUrole{default_value}{2}}, \emph{\DUrole{n}{k\_fold}\DUrole{o}{=}\DUrole{default_value}{10}}, \emph{\DUrole{n}{epochs}\DUrole{o}{=}\DUrole{default_value}{10}}, \emph{\DUrole{n}{batch\_size}\DUrole{o}{=}\DUrole{default_value}{10}}, \emph{\DUrole{n}{description}\DUrole{o}{=}\DUrole{default_value}{None}}, \emph{\DUrole{n}{save\_xxtest}\DUrole{o}{=}\DUrole{default_value}{False}}, \emph{\DUrole{n}{save\_yytest}\DUrole{o}{=}\DUrole{default_value}{False}}, \emph{\DUrole{n}{verbose}\DUrole{o}{=}\DUrole{default_value}{1}}}{}
\sphinxAtStartPar
Add tasks for an iterative manyfold \sphinxhyphen{} see \sphinxtitleref{02\sphinxhyphen{}Example\sphinxhyphen{}02.ipynb}
Generate n\_ter*k\_fold tasks, each iteration will generate a subdirectory in run\_dir.
:param settings: settings
:type settings: dict
:param run\_dir: run directoty to output k results (json files and best model)
:type run\_dir: string
:param dataset\_id: datasets id in settings file
:type dataset\_id: string
:param model\_id: model id in settings file
:type model\_id: string
:param n\_iter: number of iteration
:type n\_iter: int
:param k\_fold: number of fold
:type k\_fold: int
:param epochs: number of epochs
:type epochs: int
:param batch\_size: size of batch
:type batch\_size: int
:param description: description of the action
:type description: string
:param save\_xxtest: save x\_test as json file, or not
:type save\_xxtest: Boolean
:param save\_yytest: save y\_test and y\_pred  as json file, or not
:type save\_yytest: Boolean
:param verbose: verbosity of generated tasks
:type verbose: int
\begin{quote}\begin{description}
\item[{Returns}] \leavevmode
\sphinxAtStartPar
Nothings. Task are added to the pending taks queue.

\end{description}\end{quote}

\end{fulllineitems}

\index{add\_manyfold() (in module pyterk.task\_manager)@\spxentry{add\_manyfold()}\spxextra{in module pyterk.task\_manager}}

\begin{fulllineitems}
\phantomsection\label{\detokenize{pyterk:pyterk.task_manager.add_manyfold}}\pysiglinewithargsret{\sphinxcode{\sphinxupquote{pyterk.task\_manager.}}\sphinxbfcode{\sphinxupquote{add\_manyfold}}}{\emph{\DUrole{n}{settings}\DUrole{o}{=}\DUrole{default_value}{None}}, \emph{\DUrole{n}{run\_dir}\DUrole{o}{=}\DUrole{default_value}{None}}, \emph{\DUrole{n}{dataset\_id}\DUrole{o}{=}\DUrole{default_value}{None}}, \emph{\DUrole{n}{model\_id}\DUrole{o}{=}\DUrole{default_value}{None}}, \emph{\DUrole{n}{k\_fold}\DUrole{o}{=}\DUrole{default_value}{10}}, \emph{\DUrole{n}{epochs}\DUrole{o}{=}\DUrole{default_value}{10}}, \emph{\DUrole{n}{batch\_size}\DUrole{o}{=}\DUrole{default_value}{10}}, \emph{\DUrole{n}{description}\DUrole{o}{=}\DUrole{default_value}{None}}, \emph{\DUrole{n}{save\_xxtest}\DUrole{o}{=}\DUrole{default_value}{False}}, \emph{\DUrole{n}{save\_yytest}\DUrole{o}{=}\DUrole{default_value}{False}}, \emph{\DUrole{n}{verbose}\DUrole{o}{=}\DUrole{default_value}{1}}}{}
\sphinxAtStartPar
Add tasks for a manyfold \sphinxhyphen{} see \sphinxtitleref{01\sphinxhyphen{}Example\sphinxhyphen{}01.ipynb}
Generate k\_fold tasks, each task will generate one subdirectory in run\_dir.
:param settings: settings
:type settings: dict
:param run\_dir: run directoty to output k results (json files and best model)
:type run\_dir: string
:param dataset\_id: datasets id in settings file
:type dataset\_id: string
:param model\_id: model id in settings file
:type model\_id: string
:param k\_fold: number of fold
:type k\_fold: int
:param epochs: number of epochs
:type epochs: int
:param batch\_size: size of batch
:type batch\_size: int
:param description: description of the action
:type description: string
:param save\_xxtest: save x\_test as json file, or not
:type save\_xxtest: Boolean
:param save\_yytest: save y\_test and y\_pred  as json file, or not
:type save\_yytest: Boolean
:param verbose: verbosity of generated tasks
:type verbose: int
\begin{quote}\begin{description}
\item[{Returns}] \leavevmode
\sphinxAtStartPar
Nothings. Task are added to the pending taks queue.

\end{description}\end{quote}

\end{fulllineitems}

\index{reset() (in module pyterk.task\_manager)@\spxentry{reset()}\spxextra{in module pyterk.task\_manager}}

\begin{fulllineitems}
\phantomsection\label{\detokenize{pyterk:pyterk.task_manager.reset}}\pysiglinewithargsret{\sphinxcode{\sphinxupquote{pyterk.task\_manager.}}\sphinxbfcode{\sphinxupquote{reset}}}{}{}
\sphinxAtStartPar
Reset pending tasks. Suppress all of them !

\end{fulllineitems}

\index{run() (in module pyterk.task\_manager)@\spxentry{run()}\spxextra{in module pyterk.task\_manager}}

\begin{fulllineitems}
\phantomsection\label{\detokenize{pyterk:pyterk.task_manager.run}}\pysiglinewithargsret{\sphinxcode{\sphinxupquote{pyterk.task\_manager.}}\sphinxbfcode{\sphinxupquote{run}}}{\emph{\DUrole{n}{processes}\DUrole{o}{=}\DUrole{default_value}{None}}, \emph{\DUrole{n}{maxtasksperchild}\DUrole{o}{=}\DUrole{default_value}{10}}, \emph{\DUrole{n}{verbose}\DUrole{o}{=}\DUrole{default_value}{1}}}{}
\end{fulllineitems}

\index{seed() (in module pyterk.task\_manager)@\spxentry{seed()}\spxextra{in module pyterk.task\_manager}}

\begin{fulllineitems}
\phantomsection\label{\detokenize{pyterk:pyterk.task_manager.seed}}\pysiglinewithargsret{\sphinxcode{\sphinxupquote{pyterk.task\_manager.}}\sphinxbfcode{\sphinxupquote{seed}}}{\emph{\DUrole{n}{seed}\DUrole{o}{=}\DUrole{default_value}{None}}}{}
\sphinxAtStartPar
Init random generators with given seed

\end{fulllineitems}

\index{show\_tasks\_size() (in module pyterk.task\_manager)@\spxentry{show\_tasks\_size()}\spxextra{in module pyterk.task\_manager}}

\begin{fulllineitems}
\phantomsection\label{\detokenize{pyterk:pyterk.task_manager.show_tasks_size}}\pysiglinewithargsret{\sphinxcode{\sphinxupquote{pyterk.task\_manager.}}\sphinxbfcode{\sphinxupquote{show\_tasks\_size}}}{}{}
\sphinxAtStartPar
Print pending tasks size

\end{fulllineitems}



\section{pyterk.worker module}
\label{\detokenize{pyterk:module-pyterk.worker}}\label{\detokenize{pyterk:pyterk-worker-module}}\index{module@\spxentry{module}!pyterk.worker@\spxentry{pyterk.worker}}\index{pyterk.worker@\spxentry{pyterk.worker}!module@\spxentry{module}}
\sphinxAtStartPar
This module is for internal use only \sphinxhyphen{} You do not have to interact with ;\sphinxhyphen{}).
\index{get\_model\_family() (in module pyterk.worker)@\spxentry{get\_model\_family()}\spxextra{in module pyterk.worker}}

\begin{fulllineitems}
\phantomsection\label{\detokenize{pyterk:pyterk.worker.get_model_family}}\pysiglinewithargsret{\sphinxcode{\sphinxupquote{pyterk.worker.}}\sphinxbfcode{\sphinxupquote{get\_model\_family}}}{\emph{\DUrole{n}{model}}}{}
\sphinxAtStartPar
Should return the model family : ‘tensorflow’, ‘keras’ or ‘sklearn’

\end{fulllineitems}

\index{init() (in module pyterk.worker)@\spxentry{init()}\spxextra{in module pyterk.worker}}

\begin{fulllineitems}
\phantomsection\label{\detokenize{pyterk:pyterk.worker.init}}\pysiglinewithargsret{\sphinxcode{\sphinxupquote{pyterk.worker.}}\sphinxbfcode{\sphinxupquote{init}}}{\emph{\DUrole{n}{s}}, \emph{\DUrole{n}{l}}, \emph{\DUrole{n}{v}}}{}
\end{fulllineitems}

\index{model\_fit() (in module pyterk.worker)@\spxentry{model\_fit()}\spxextra{in module pyterk.worker}}

\begin{fulllineitems}
\phantomsection\label{\detokenize{pyterk:pyterk.worker.model_fit}}\pysiglinewithargsret{\sphinxcode{\sphinxupquote{pyterk.worker.}}\sphinxbfcode{\sphinxupquote{model\_fit}}}{\emph{\DUrole{n}{run\_dir}\DUrole{o}{=}\DUrole{default_value}{None}}, \emph{\DUrole{n}{dataset\_id}\DUrole{o}{=}\DUrole{default_value}{None}}, \emph{\DUrole{n}{train\_index}\DUrole{o}{=}\DUrole{default_value}{None}}, \emph{\DUrole{n}{test\_index}\DUrole{o}{=}\DUrole{default_value}{None}}, \emph{\DUrole{n}{model\_id}\DUrole{o}{=}\DUrole{default_value}{None}}, \emph{\DUrole{n}{epochs}\DUrole{o}{=}\DUrole{default_value}{None}}, \emph{\DUrole{n}{batch\_size}\DUrole{o}{=}\DUrole{default_value}{None}}, \emph{\DUrole{n}{seed}\DUrole{o}{=}\DUrole{default_value}{None}}, \emph{\DUrole{n}{description}\DUrole{o}{=}\DUrole{default_value}{None}}, \emph{\DUrole{n}{save\_xxtest}\DUrole{o}{=}\DUrole{default_value}{False}}, \emph{\DUrole{n}{save\_yytest}\DUrole{o}{=}\DUrole{default_value}{False}}}{}
\end{fulllineitems}

\index{model\_fit\_sklearn() (in module pyterk.worker)@\spxentry{model\_fit\_sklearn()}\spxextra{in module pyterk.worker}}

\begin{fulllineitems}
\phantomsection\label{\detokenize{pyterk:pyterk.worker.model_fit_sklearn}}\pysiglinewithargsret{\sphinxcode{\sphinxupquote{pyterk.worker.}}\sphinxbfcode{\sphinxupquote{model\_fit\_sklearn}}}{\emph{\DUrole{n}{model}}, \emph{\DUrole{n}{run\_dir}\DUrole{o}{=}\DUrole{default_value}{None}}, \emph{\DUrole{n}{x\_train}\DUrole{o}{=}\DUrole{default_value}{None}}, \emph{\DUrole{n}{y\_train}\DUrole{o}{=}\DUrole{default_value}{None}}, \emph{\DUrole{n}{x\_test}\DUrole{o}{=}\DUrole{default_value}{None}}, \emph{\DUrole{n}{y\_test}\DUrole{o}{=}\DUrole{default_value}{None}}, \emph{\DUrole{n}{save\_xxtest}\DUrole{o}{=}\DUrole{default_value}{False}}, \emph{\DUrole{n}{save\_yytest}\DUrole{o}{=}\DUrole{default_value}{False}}}{}
\end{fulllineitems}

\index{model\_fit\_tensorflow() (in module pyterk.worker)@\spxentry{model\_fit\_tensorflow()}\spxextra{in module pyterk.worker}}

\begin{fulllineitems}
\phantomsection\label{\detokenize{pyterk:pyterk.worker.model_fit_tensorflow}}\pysiglinewithargsret{\sphinxcode{\sphinxupquote{pyterk.worker.}}\sphinxbfcode{\sphinxupquote{model\_fit\_tensorflow}}}{\emph{\DUrole{n}{model}}, \emph{\DUrole{n}{run\_dir}\DUrole{o}{=}\DUrole{default_value}{None}}, \emph{\DUrole{n}{x\_train}\DUrole{o}{=}\DUrole{default_value}{None}}, \emph{\DUrole{n}{y\_train}\DUrole{o}{=}\DUrole{default_value}{None}}, \emph{\DUrole{n}{x\_test}\DUrole{o}{=}\DUrole{default_value}{None}}, \emph{\DUrole{n}{y\_test}\DUrole{o}{=}\DUrole{default_value}{None}}, \emph{\DUrole{n}{epochs}\DUrole{o}{=}\DUrole{default_value}{None}}, \emph{\DUrole{n}{batch\_size}\DUrole{o}{=}\DUrole{default_value}{None}}, \emph{\DUrole{n}{save\_xxtest}\DUrole{o}{=}\DUrole{default_value}{False}}, \emph{\DUrole{n}{save\_yytest}\DUrole{o}{=}\DUrole{default_value}{False}}}{}
\end{fulllineitems}



\chapter{MultipleRegression module}
\label{\detokenize{MultipleRegression:module-MultipleRegression}}\label{\detokenize{MultipleRegression:multipleregression-module}}\label{\detokenize{MultipleRegression::doc}}\index{module@\spxentry{module}!MultipleRegression@\spxentry{MultipleRegression}}\index{MultipleRegression@\spxentry{MultipleRegression}!module@\spxentry{module}}
\sphinxAtStartPar
Module to train Multiple Linear Regression with Scheffe ineteraction terms with iterative k\sphinxhyphen{}fold crossvalidation

\sphinxAtStartPar
Contains functions to :
\begin{itemize}
\item {} 
\sphinxAtStartPar
generate interactions

\item {} 
\sphinxAtStartPar
train regression models

\item {} 
\sphinxAtStartPar
plot iterative k\sphinxhyphen{}fold crossvalidation results

\end{itemize}
\index{Scheffe\_interactions\_terms() (in module MultipleRegression)@\spxentry{Scheffe\_interactions\_terms()}\spxextra{in module MultipleRegression}}

\begin{fulllineitems}
\phantomsection\label{\detokenize{MultipleRegression:MultipleRegression.Scheffe_interactions_terms}}\pysiglinewithargsret{\sphinxcode{\sphinxupquote{MultipleRegression.}}\sphinxbfcode{\sphinxupquote{Scheffe\_interactions\_terms}}}{\emph{\DUrole{n}{data}}, \emph{\DUrole{n}{in\_percent}\DUrole{o}{=}\DUrole{default_value}{\textquotesingle{}True\textquotesingle{}}}, \emph{\DUrole{n}{compo\_columns}\DUrole{o}{=}\DUrole{default_value}{{[}\textquotesingle{}Zr\textquotesingle{}, \textquotesingle{}Nb\textquotesingle{}, \textquotesingle{}Mo\textquotesingle{}, \textquotesingle{}Ti\textquotesingle{}, \textquotesingle{}Cr\textquotesingle{}{]}}}}{}
\sphinxAtStartPar
Shaping composition in percentage rate into percentage
Compute interaction terms for all Scheffe interactions for quartic multiple regression and add it to dataframe data
\begin{quote}\begin{description}
\item[{Parameters}] \leavevmode
\sphinxAtStartPar
\sphinxstyleliteralstrong{\sphinxupquote{panda.DataFrame}} \textendash{} dataset that contains compositions in Zr, Nb, Mo, Ti, Cr in columns of the same name

\item[{Returns}] \leavevmode
\sphinxAtStartPar
extended input dataset with interactions

\item[{Return type}] \leavevmode
\sphinxAtStartPar
DataFrame

\end{description}\end{quote}

\end{fulllineitems}

\index{fit\_outputs() (in module MultipleRegression)@\spxentry{fit\_outputs()}\spxextra{in module MultipleRegression}}

\begin{fulllineitems}
\phantomsection\label{\detokenize{MultipleRegression:MultipleRegression.fit_outputs}}\pysiglinewithargsret{\sphinxcode{\sphinxupquote{MultipleRegression.}}\sphinxbfcode{\sphinxupquote{fit\_outputs}}}{\emph{\DUrole{n}{model\_expression}}, \emph{\DUrole{n}{k}}, \emph{\DUrole{n}{nb\_it}}, \emph{\DUrole{n}{output}}, \emph{\DUrole{n}{X}}, \emph{\DUrole{n}{y}}}{}
\sphinxAtStartPar
Takes an OLS model expression, and use it to perform regression between X and y .
Model regression is performed using iterative k\sphinxhyphen{}fold crossvalidation
Evaluation is performed through R2 and MAE computation
\begin{quote}\begin{description}
\item[{Parameters}] \leavevmode\begin{itemize}
\item {} 
\sphinxAtStartPar
\sphinxstyleliteralstrong{\sphinxupquote{OLS\sphinxhyphen{}formula}} \textendash{} contain OLS formula for regression

\item {} 
\sphinxAtStartPar
\sphinxstyleliteralstrong{\sphinxupquote{k}} (\sphinxstyleliteralemphasis{\sphinxupquote{int}}) \textendash{} number of folds for iterative k\sphinxhyphen{}fold crossfvalidation

\item {} 
\sphinxAtStartPar
\sphinxstyleliteralstrong{\sphinxupquote{nb\_it}} (\sphinxstyleliteralemphasis{\sphinxupquote{int}}) \textendash{} number of iterations for iterative k\sphinxhyphen{}fold crossfvalidation

\item {} 
\sphinxAtStartPar
\sphinxstyleliteralstrong{\sphinxupquote{output}} (\sphinxstyleliteralemphasis{\sphinxupquote{str}}) \textendash{} name of the Y output to fit

\item {} 
\sphinxAtStartPar
\sphinxstyleliteralstrong{\sphinxupquote{X}} (\sphinxstyleliteralemphasis{\sphinxupquote{panda.DtataFrame}}) \textendash{} contains composition and interaction terms for regression input

\item {} 
\sphinxAtStartPar
\sphinxstyleliteralstrong{\sphinxupquote{y}} (\sphinxstyleliteralemphasis{\sphinxupquote{panda.DataFrame}}) \textendash{} contains single column dataframe with regression output

\end{itemize}

\item[{Return model}] \leavevmode
\sphinxAtStartPar
model coefficients and p\sphinxhyphen{}values

\item[{Return type}] \leavevmode
\sphinxAtStartPar
statsmodels.regression.linear\_model.RegressionResultsWrapper

\item[{Return MAE\_list}] \leavevmode
\sphinxAtStartPar
list of MAE for every run of iterative k\sphinxhyphen{}fold crossvalidation, between expected vs predicted value on test set

\item[{Return type}] \leavevmode
\sphinxAtStartPar
list

\item[{Return R2\_list}] \leavevmode
\sphinxAtStartPar
list of R2 for every run of iterative k\sphinxhyphen{}fold crossvalidation, between expected vs predicted value on test set

\item[{Return type}] \leavevmode
\sphinxAtStartPar
list

\item[{Returns}] \leavevmode
\sphinxAtStartPar
Y\_pred : list of predicted values on test set

\item[{Return type}] \leavevmode
\sphinxAtStartPar
list

\item[{Returns}] \leavevmode
\sphinxAtStartPar
Y\_test : list of expected values on test set

\item[{Return type}] \leavevmode
\sphinxAtStartPar
list

\end{description}\end{quote}

\end{fulllineitems}

\index{plot\_result() (in module MultipleRegression)@\spxentry{plot\_result()}\spxextra{in module MultipleRegression}}

\begin{fulllineitems}
\phantomsection\label{\detokenize{MultipleRegression:MultipleRegression.plot_result}}\pysiglinewithargsret{\sphinxcode{\sphinxupquote{MultipleRegression.}}\sphinxbfcode{\sphinxupquote{plot\_result}}}{\emph{\DUrole{n}{metric}}, \emph{\DUrole{n}{output}}, \emph{\DUrole{n}{val\_metric}}, \emph{\DUrole{n}{Y\_pred}}, \emph{\DUrole{n}{Y\_test}}, \emph{\DUrole{n}{min\_hist}}, \emph{\DUrole{n}{max\_hist}}, \emph{\DUrole{n}{iter}}, \emph{\DUrole{n}{kfold}}, \emph{\DUrole{n}{save\_distri}}, \emph{\DUrole{n}{save\_regression}}}{}
\sphinxAtStartPar
Plot metric histogram and regression between predictions and test values and save graphs
\begin{quote}\begin{description}
\item[{Parameters}] \leavevmode\begin{itemize}
\item {} 
\sphinxAtStartPar
\sphinxstyleliteralstrong{\sphinxupquote{metric}} (\sphinxstyleliteralemphasis{\sphinxupquote{str}}) \textendash{} name of the metric distribution to plot

\item {} 
\sphinxAtStartPar
\sphinxstyleliteralstrong{\sphinxupquote{output}} (\sphinxstyleliteralemphasis{\sphinxupquote{str}}) \textendash{} name of the Y output to fit

\item {} 
\sphinxAtStartPar
\sphinxstyleliteralstrong{\sphinxupquote{val\_metric}} (\sphinxstyleliteralemphasis{\sphinxupquote{list}}) \textendash{} list of MAE for every run of iterative k\sphinxhyphen{}fold crossvalidation, between expected vs predicted value on test set

\item {} 
\sphinxAtStartPar
\sphinxstyleliteralstrong{\sphinxupquote{Y\_pred}} (\sphinxstyleliteralemphasis{\sphinxupquote{list}}) \textendash{} list of predicted values on test set

\item {} 
\sphinxAtStartPar
\sphinxstyleliteralstrong{\sphinxupquote{Y\_test}} (\sphinxstyleliteralemphasis{\sphinxupquote{list}}) \textendash{} list of expected values on test set

\item {} 
\sphinxAtStartPar
\sphinxstyleliteralstrong{\sphinxupquote{min\_hist}} (\sphinxstyleliteralemphasis{\sphinxupquote{int}}) \textendash{} minimun of abscissa for metric distribution histogram

\item {} 
\sphinxAtStartPar
\sphinxstyleliteralstrong{\sphinxupquote{max\_hist}} (\sphinxstyleliteralemphasis{\sphinxupquote{int}}) \textendash{} maximum of abscissa for metric distribution histogram

\item {} 
\sphinxAtStartPar
\sphinxstyleliteralstrong{\sphinxupquote{iter}} (\sphinxstyleliteralemphasis{\sphinxupquote{int}}) \textendash{} plot regression over a certain number of iterations

\item {} 
\sphinxAtStartPar
\sphinxstyleliteralstrong{\sphinxupquote{save\_distri}} (\sphinxstyleliteralemphasis{\sphinxupquote{str}}) \textendash{} path to save metric distribution

\item {} 
\sphinxAtStartPar
\sphinxstyleliteralstrong{\sphinxupquote{save\_regression}} (\sphinxstyleliteralemphasis{\sphinxupquote{std}}) \textendash{} path to save regression

\end{itemize}

\item[{Parm int kfold}] \leavevmode
\sphinxAtStartPar
plot regression over a certain number of k\sphinxhyphen{}fold for each iteration

\end{description}\end{quote}

\end{fulllineitems}



\chapter{Indices and tables}
\label{\detokenize{index:indices-and-tables}}\begin{itemize}
\item {} 
\sphinxAtStartPar
\DUrole{xref,std,std-ref}{genindex}

\item {} 
\sphinxAtStartPar
\DUrole{xref,std,std-ref}{modindex}

\item {} 
\sphinxAtStartPar
\DUrole{xref,std,std-ref}{search}

\end{itemize}


\renewcommand{\indexname}{Python Module Index}
\begin{sphinxtheindex}
\let\bigletter\sphinxstyleindexlettergroup
\bigletter{m}
\item\relax\sphinxstyleindexentry{MultipleRegression}\sphinxstyleindexpageref{MultipleRegression:\detokenize{module-MultipleRegression}}
\indexspace
\bigletter{p}
\item\relax\sphinxstyleindexentry{pyterk}\sphinxstyleindexpageref{pyterk:\detokenize{module-pyterk}}
\item\relax\sphinxstyleindexentry{pyterk.config}\sphinxstyleindexpageref{pyterk:\detokenize{module-pyterk.config}}
\item\relax\sphinxstyleindexentry{pyterk.models}\sphinxstyleindexpageref{pyterk:\detokenize{module-pyterk.models}}
\item\relax\sphinxstyleindexentry{pyterk.reporter}\sphinxstyleindexpageref{pyterk:\detokenize{module-pyterk.reporter}}
\item\relax\sphinxstyleindexentry{pyterk.task\_manager}\sphinxstyleindexpageref{pyterk:\detokenize{module-pyterk.task_manager}}
\item\relax\sphinxstyleindexentry{pyterk.worker}\sphinxstyleindexpageref{pyterk:\detokenize{module-pyterk.worker}}
\end{sphinxtheindex}
\renewcommand{\indexname}{MATLAB Module Index}
\begin{sphinxtheindex}
\let\bigletter\sphinxstyleindexlettergroup
\bigletter{m}
\item\relax\sphinxstyleindexentry{modules}\sphinxstyleindexpageref{ExperimentsPlannification:\detokenize{module-modules}}
\end{sphinxtheindex}

\renewcommand{\indexname}{Index}
\printindex
\end{document}